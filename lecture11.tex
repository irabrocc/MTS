% 声明为子文件,指定主文件
\documentclass[main.tex]{subfiles}

\begin{document}
\pagestyle{plain}
\setcounter{chapter}{10}

\chapter{Lecture11}
\label{chap:chapter11}

\par \noindent \textbf{Exercise} Let $U_{\epsilon}(x) = \{y\in X| d(x, y) < \epsilon\}$. Prove that the collection of all such $U_{\epsilon}(x)$ forms a basis for a topology on $X$. 
\begin{definition}
    A topological space $(X, \mathcal{T})$ is metrizable if there exists a metric $d$ on $X$ such that the topology induced by $d$ is equal to $\mathcal{T}$.
\end{definition} 

\begin{lemma}
    Let $d$ and $d'$ be two metrics on $X$ defining the topologies $\mathcal{T}$ and $\mathcal{T}'$ respectively. Then $\mathcal{T}'$ is finer than $\mathcal{T}$ if and only if for each $x \in X$ and each $\epsilon > 0$, there exists $\delta > 0$ such that 
    \begin{equation}
        U_{\delta}^{d'}(x) \subseteq U_{\epsilon}^{d}(x)
    \end{equation}
\end{lemma}

\begin{property}
    The metrics $d_1, d_2, d_{\infty}$ define the same topology on $\mathbb{R}^n$. This is the product topology(or box as $n$ is finite) on $\mathbb{R}^n = \mathbb{R} \times \mathbb{R} \times \cdots \times \mathbb{R}$ ($n$ times). 
\end{property}
\par \noindent \textbf{Proof} This can be shown using graphs. 
\par In the following, we will define a metric that is used to bound another metric. Let $d$ be a metric on $X$. Define
\begin{equation}
    \bar{d}(x, y) = \min\{d(x, y), 1\}
\end{equation}
\begin{theorem}
    $\bar{d}$ is a metric on $X$ defining the same topology as $d$.
\end{theorem}
\par First, we need to check that $\bar{d}$ is a metric. The only non-trivial part is the triangle inequality. For any $x, y, z \in X$, we want to show that
\begin{equation}
    \bar{d}(x, z) \leq \bar{d}(x, y) + \bar{d}(y, z)
\end{equation}
If $\bar{d}(x, y)$ or $\bar{d}(y, z)$ equals 1, then the right hand side is at least 1, and the inequality holds. If both $\bar{d}(x, y)$ and $\bar{d}(y, z)$ are less than 1, then $\bar{d}(x, y) = d(x, y)$ and $\bar{d}(y, z) = d(y, z)$. By the triangle inequality of $d$, we have
\begin{equation}
    \bar{d}(x, z) = \min\{d(x, z), 1\} \leq d(x, z) \leq d(x, y) + d(y, z) = \bar{d}(x, y) + \bar{d}(y, z)
\end{equation}
\par Now we note that in any metric space, the collection of $\epsilon$-balls with $\epsilon < 1$ forms a basis for the topology induced by the metric. It follows that $\bar{d}$ and $d$ induce the same topology on $X$, because the collections of $\epsilon$-balls with $\epsilon < 1$ are the same for both metrics.
\mbox{} \\ \null \hfill $\blacksquare$ 

\par We want to define a metric on $\mathbb{R}^{\omega}$. 
\begin{definition}
    For $x = (x_1, x_2, \ldots), y = (y_1, y_2, \ldots) \in \mathbb{R}^{\omega}$, define
    \begin{equation}
        \bar{d}(x, y) = \min(\sup(|x_i - y_i|), 1)
    \end{equation}
    which is called the uniform metric on $\mathbb{R}^{\omega}$. And the topology it induces is called the uniform topology.
\end{definition}
Also we can generalize the metric for $\mathbb{R}^{J}$ for any $J$. (sup(min) or min(sup) does not matter)
\begin{theorem}
    Uniform topology on $\mathbb{R}^{J}$ is: 
    \begin{itemize}
        \item (strictly) finer than the direct product topology;
        \item (strictly) coarser than the box topology. 
    \end{itemize}
\end{theorem}
provided that $J$ is infinite.
\par \noindent \textbf{Proof} 
\par First, we show that the uniform topology is finer than the product topology. Let $x \in \mathbb{R}^{J}$ and let $U$ be a basic open set in the product topology containing $x$. Then
\begin{equation}
    U = \prod_{\alpha \in J} U_{\alpha}
\end{equation}
where $U_{\alpha}$ is an open set in $\mathbb{R}$ and $U_{\alpha} = \mathbb{R}$ for all but finitely many $\alpha$. Let $K$ be the finite set of indices $\alpha$ such that $U_{\alpha} \neq \mathbb{R}$. For each $\alpha \in K$, there exists $1 > \epsilon_{\alpha} > 0$ such that
\begin{equation}
    U_{\epsilon_{\alpha}}(x_{\alpha}) \subseteq U_{\alpha}
\end{equation}
Let $\epsilon = \min_{\alpha \in K} \epsilon_{\alpha}$. Then the open ball $U_{\epsilon}^{\bar{d}}(x)$ in the uniform topology is contained in $U$. Thus the uniform topology is finer than the product topology. 
\par On the other hand, let 
\begin{equation}
    U = \prod_{\alpha \in J} (x_{\alpha} - \epsilon, x_{\alpha} + \epsilon)
\end{equation}
with $\epsilon < 1$ which is a basis element in the uniform topology. And $U$ is also a basis element in the box topology. Thus the uniform topology is coarser than the box topology.
\par For the strictness of the two sides, we can use some examples. 
\par Since the uniform topology is finer than the product topology and the the box topology is finer than the uniform topology, we have that for a sequence $(x_n)$ in $\mathbb{R}^{J}$, if $x_n \rightarrow x$ in the box topology, then $x_n \rightarrow x$ in the uniform topology, and if $x_n \rightarrow x$ in the uniform topology, then $x_n \rightarrow x$ in the product topology. 
\par So for the strictness, it suffices to find a sequence that converges in one topology but not in the other. 
\par Let $\{w_i\}_{i=1}^{\infty}$ be a sequence in $\mathbb{R}^{\omega}$ defined as 
\begin{equation}
    w_1 = (1, 1, 1, \ldots), w_2 = (0, 2, 2, \ldots), w_3 = (0, 0, 3, 3, \ldots), \ldots
\end{equation}
Then $w_n \rightarrow 0$ in the product topology. But it diverges in the uniform topology and the box topology. Thus the uniform topology is strictly finer than the product topology. 
\par Let $\{x_i\}_{i=1}^{\infty}$ be a sequence in $\mathbb{R}^{\omega}$ defined as
\begin{equation}
    x_1 = (1, 1, 1, \ldots), x_2 = (0, \frac{1}{2}, \frac{1}{2}, \ldots), x_3 = (0, 0, \frac{1}{3}, \frac{1}{3}, \ldots), \ldots
\end{equation}
\par Then w.l.o.g. if $\epsilon < 1$, the ball in the uniform topology 
\begin{equation}
    U_{\epsilon}^{\bar{d}}(0) = \{y \in \mathbb{R}^{\omega} | \sup(|y_i - 0|) < \epsilon\} = \{y \in \mathbb{R}^{\omega} | |y_i| < \epsilon, \forall i\}
\end{equation}
contains all but finitely many $x_n$. Thus $x_n \rightarrow 0$ in the uniform topology. However, for the box topology, consider 
\begin{equation}
    (-1, 1) \times (-\frac{1}{2}, \frac{1}{2}) \times (-\frac{1}{3}, \frac{1}{3}) \times \cdots
\end{equation}
which is a basis element in the box topology containing $0$. This set does not contain any $x_n$ for all $n$. Thus $x_n$ does not converge to $0$ in the box topology. Therefore, the uniform topology is strictly coarser than the box topology. 
\par Consider the sequence $\{y_n\}_{n=1}^{\infty}$ in $\mathbb{R}^{\omega}$ defined as 
\begin{equation}
    y_1 = (1, 0, 0, 0, \ldots), y_2 = (\frac{1}{2}, \frac{1}{2}, 0, 0, \ldots), y_3 = (\frac{1}{3}, \frac{1}{3}, \frac{1}{3}, 0, 0, \ldots), \ldots
\end{equation}
\par Then $y_n \rightarrow 0$ in all three topologies.
\mbox{} \\ \null \hfill $\blacksquare$ 
\par Recall that for two metric spaces $(X, d)$ and $(Y, d')$, a function $f: X \rightarrow Y$ is continuous if and only if for each $x \in X$ and each $\epsilon > 0$, there exists $\delta > 0$ such that
\begin{equation}
    d(x, y) < \delta \Rightarrow d'(f(x), f(y)) < \epsilon
\end{equation}

\par \noindent \textbf{Problem} If $x_n \rightarrow 0$ in the box topology on $\mathbb{R}^{\omega}$, what can be said about $x_n$? Is the guess: $\exists M, \forall m \ge M, x_n^{(m)} = 0$ correct? 
\par \noindent \textbf{Answer} Not sure. 

\begin{theorem}
    [The sequence lemma] 
    Let $X$ be a topological space, $A \subseteq X$. If there is a sequence of points $x_n \in A$ with $\lim_{n \rightarrow \infty} x_n = x$, then $x \in \bar{A}$. The converse holds if $X$ is metrizable.
\end{theorem}

\par \noindent \textbf{Proof} If $x_n \in A$ and $\lim_{n \rightarrow \infty} x_n = x$, then for each open set $U$ containing $x$, we have $U\cap A$ contains all but finitely many $x_n$, so $U \cap A \neq \emptyset$. Thus $x \in \bar{A}$.
\par For the converse, suppose $X$ is metrizable with metric $d$. If $x \in \bar{A}$, then for each $n \in \mathbb{N}$, the open ball $U_{1/n}(x)$ intersects $A$. Thus we can choose $x_n \in U_{1/n}(x) \cap A$. Then $\lim_{n \rightarrow \infty} x_n = x$($\forall \epsilon > 0, \exists N > \frac{1}{\epsilon}$, such that $d(x_n, x) < \epsilon$ for all $n > N$). 

\mbox{} \\ \null \hfill $\blacksquare$ 

\end{document}