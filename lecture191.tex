% 声明为子文件,指定主文件
\documentclass[main.tex]{subfiles}

\begin{document}
\pagestyle{plain}
\setcounter{chapter}{18}

\chapter{Later}
\label{chap:chapter19}

\begin{theorem}
    [Urysohn's Lemma] 
    Let $X$ be a normal space. If $A$ and $B$ are disjoint closed subsets of $X$, then there exists a continuous function $f: X \to [0, 1]$ such that $f(a) = 0$ for all $a \in A$ and $f(b) = 1$ for all $b \in B$.
\end{theorem}

\begin{definition}
    Suppose $X$ satisfies (T1). If a point and a closed set can be separated by a continuous function as in Urysohn's lemma, then $X$ is called completely regular(CR). (Sometimes it is called $T_{3\frac{1}{2}}$ space.)
\end{definition}

\par \noindent \textbf{Remark} 
\par Normal $\Rightarrow$ CR $\Rightarrow$ Regular. That's why we call it $T_{3\frac{1}{2}}$ space.

\begin{property}
    If $X$ is completely regular, then it is regular. 
\end{property}

\par \noindent \textbf{Proof} 
\par Take $x_0\in X$ and a closed set $A$ such that $x_0 \notin A$. Since $X$ is CR, there exists a continuous function $f: X \to [0, 1]$ such that $f(x_0) = 0$ and $f(a) = 1$ for all $a \in A$. Let $U = f^{-1}([0, \frac{1}{2}))$ and $V = f^{-1}((\frac{1}{2}, 1])$. Then $U$ and $V$ are open sets because $f$ is continuous and $[0, \frac{1}{2})$, $(\frac{1}{2}, 1]$ are open in $[0, 1]$. So $x_0 \in U$, $A \subseteq V$ and $U \cap V = \varnothing$. Thus $X$ is regular.
\mbox{} \\ \null \hfill $\blacksquare$ 

\begin{theorem}
    \begin{enumerate}
        \item Subspace of CR space is CR. 
        \item If $X_{\alpha}$ is CR, then $\prod X_{\alpha}$ with product topology is CR. 
    \end{enumerate}
\end{theorem}
\par \noindent \textbf{Proof} 
\par Let $X$ be CR, and $Y \subseteq X$ with subspace topology. Take $y_0 \in Y$ and a closed set $A \subseteq Y$ such that $y_0 \notin A$. Let $\bar{A}$ be the closure of $A$ in $X$. Then $\bar{A}$ is closed in $X$ and $y_0 \notin \bar{A}$. Since $X$ is CR, there exists a continuous function $f: X \to [0, 1]$ such that $f(y_0) = 0$ and $f(a) = 1$ for all $a \in \bar{A}$. Restrict $f$ to $Y$, we get a continuous function $f|_Y: Y \to [0, 1]$ such that $f|_Y(y_0) = 0$ and $f|_Y(a) = 1$ for all $a \in A$. Thus $Y$ is CR.
\par Take $A \subseteq \prod X_{\alpha}$ closed and $x = (x_{\alpha}) \notin A$. Then there exists a basic open set $U = \prod U_{\alpha}$ such that $x \in U$ and $U \cap A = \emptyset$, where $U_{\alpha}$ is open in $X_{\alpha}$ and $U_{\alpha} = X_{\alpha}$ for all but finitely many $\alpha$. Let those finitely many $\alpha$ be $\alpha_1, \alpha_2, \ldots, \alpha_n$. Since $X_{\alpha_i}$ is CR, there exists a continuous function $f_{\alpha_i}: X_{\alpha_i} \to [0, 1]$ such that $f_{\alpha_i}(x_{\alpha_i}) = 0$ and $f_{\alpha_i}(a) = 1$ for all $a \in X_{\alpha_i} \setminus U_{\alpha_i}$. So we can define a continuous function $f: \prod X_{\alpha} \to [0, 1]$ by
\[
    f(x) = \prod_{i = 1}^{n} f_{\alpha_i}(x_{\alpha_i}). 
\]
Then $f(x) = 0$ and for all $a \in A$, $f(a) = 1$. Thus $\prod X_{\alpha}$ is CR.
(To be checked)
\mbox{} \\ \null \hfill $\blacksquare$  

\begin{example}
    $\mathbb{R}_l$ is normal then it is CR. So $\mathbb{R}_l^2$ is CR. But it is not normal as we have shown. 
\end{example}

\par \noindent \textbf{Fact}(no proof) There exist regular spaces that are not CR. 

\par \noindent \textbf{Remark} Proof of Urysohn's lemma for metric spaces(exercise): 
\par Let $X$ be metric space with metric $d: X \times X \to \mathbb{R}$. Given two disjoint closed sets $A, B \subseteq X$, define $d_B(x) = d(x, B)$ which vanishes exactly on $B$. Then define $d_A(x) = d(x, A)$ which vanishes exactly on $A$. Now define 
\[
    f(x) = \frac{d_A(x)}{d_A(x) + d_B(x)}.
\]
Then $f(A) = \{0\}$ and $f(B) = \{1\}$ and $f$ is continuous because we've proved that distance from a point to a closed set is continuous.   

\begin{theorem}
    [Urysohn Metrization Theorem] 
    Every regular space with contable basis is metrizable. 
\end{theorem}
\par \noindent \textbf{Idea of Proof} Embede $X$ into $[0, 1]^{\mathbb{N}} \subseteq \mathbb{R}^{\mathbb{N}}$. This $\mathbb{R}^{\mathbb{N}}$ can be with product topology or uniform metric topology.  
\par \noindent \textbf{Step 1} Construct a family $f_n: X \rightarrow [0, 1]$ of functions such that $\forall x_0\in X$ and $\forall$ neighborhood $U$ of $x_0$, there exists $n$ such that $f_n(x_0) > 1$ and $f_n(x) = 0$ for all $x \notin U$. 
\par For each $x_0$ and neighborhood $U$ of $x_0$, since $X$ is regular, by Urysohn's lemma, there exists such functions. But we want to get countably many of them. 
\par Given $x_0\in U$ there exists $B_m $ such that $x_0 \in B_m \subseteq U$. Since $X$ is regular, there exists $B_n$ such that $x_0 \in B_n$ and $\bar{B_n} \subseteq B_m$. By Urysohn's lemma, there exist

\end{document}