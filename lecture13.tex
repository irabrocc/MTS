% 声明为子文件,指定主文件
\documentclass[main.tex]{subfiles}

\begin{document}
\pagestyle{plain}
\setcounter{chapter}{12}

\chapter{Later}
\label{chap:chapter13}

\section{Metrizability}
\par Recall the following theorem. 
\begin{theorem}
    [Sequence Lemma] 
    Let $X$ be a topological space. Let $A \subseteq X$. If $\exists \{x_n\} \subseteq A$ such that $x_n \to x$ in $X$, then $x \in \overline{A}$. The converse is true if $X$ is metrizable. 
\end{theorem}

\begin{theorem}
    [Heine's definition of limit] 
    Let $f: X \rightarrow Y$. If $f$ is continuous, then for every convergent sequence $x_n \to x$ in $X$, we have $f(x_n) \to f(x)$ in $Y$. The converse is true if $X$ is metrizable.
\end{theorem}
\par \noindent \textbf{Poof}
\par ($\Rightarrow$) Let $f$ be continuous. Let $V\ni f(x)$ be open in $Y$. Then $f^{-1}(V)$ is open in $X$ and contains $x$. Since $x_n \to x$, $\exists N$ such that $\forall n \geq N$, $x_n \in f^{-1}(V)$. Thus, $\forall n \geq N$, $f(x_n) \in V$. Hence, $f(x_n) \to f(x)$.
\par ($\Leftarrow$) Recall that $f$ is continuous if and only if for all $A \subseteq X$, $f(\overline{A}) \subseteq \overline{f(A)}$. 
\par Let $A \subseteq X$. Let $x\in \bar{A}$. Since $X$ is metrizable, by the sequence lemma, there exists a sequence $\{x_n\} \subseteq A$ such that $x_n \to x$. By the assumption, $f(x_n) \to f(x)$. Thus, by the sequence lemma again, $f(x) \in \overline{f(A)}$. So $f(\overline{A}) \subseteq \overline{f(A)}$. Hence, $f$ is continuous. 
\mbox{} \\ \null \hfill $\blacksquare$

\begin{definition}
    $X$ is first-countable if it has a countable basis at each $x\in X$. Given $x \in X$, there exists a countable collection of open sets $\{U_n\}$ such that for any open set $U$ containing $x$, $\exists n$ such that $U_n \subseteq U$. (From this we can construct $\tilde{U}_n = \bigcap_{i=1}^{n} U_i$ such that $\{\tilde{U}_n\}$ is also a countable basis at $x$ with $\tilde{U}_{n + 1} \subseteq \tilde{U_n}$.)
\end{definition}

\begin{definition}
    $X$ is second-countable if it has a countable basis for the topology. There exists countable a basis $\mathcal{B}$ for $X$ such that $\forall x\in X, \forall U$ open in $X$ containing $x$, $\exists B \in \mathcal{B}$ such that $x \in B \subseteq U$. 
\end{definition}

\begin{property}
    If $X$ is second-countable, then $X$ is first-countable.
\end{property}

\begin{example}
    $\mathbb{R}^n$ has a countable basis. For instance, the set of all open balls with rational radii and centers at points with rational coordinates forms a countable basis for the standard topology on $\mathbb{R}^n$. 
\end{example}

\begin{example}
    $\mathbb{R}$ with finite complement topology is not first-countable. Let $U_1, U_2, \ldots$ be a countable open sets containing $x$. Then $\bigcap_{n=1}^{\infty} U_n \setminus \{x\}$ is not empty, so there exists $y \neq x$ such that $y \in U_n$ for all $n$. Take $U = \mathbb{R} \setminus \{y\}$, which is open and contains $x$. However, there is no $U_n$ such that $U_n \subseteq U$. Hence, $\mathbb{R}$ with finite complement topology is not first-countable.
\end{example}

\begin{example}
    $X$ is uncountable with discrete topology. Then $\forall x\in X$, the set $\{x\} $ is open. So any basis of $X$ must contain $\{x\}$ for all $x\in X$. So $X$ is not second-countable. But $X$ is metrizable thus first-countable. 
\end{example}

\begin{example}
    $\mathbb{R}^2$ with "Amazon River metric". Define 
    \begin{equation}
        d((x, y), (x', y')) = \begin{cases}
            |y - y'|, & x = x' \\ 
            |y| + |y'| + |x - x'|, & x \neq x'
    \end{cases}
    \end{equation}
    Then $\{(x, y)| x = x_0, y\in (y_0 - \epsilon, y_0 + \epsilon)\}$ with $\epsilon < |y_0|$ is an open ball centered at $(x_0, y_0)$. There are uncountable many such disjoint open sets. So $\mathbb{R}^2$ with Amazon River metric is not second-countable. But it is metrizable thus first-countable. 
\end{example}
\begin{example}
    Let $\mathbb{R}_l $ be the set of real numbers with the lower limit topology. Then $\mathbb{R}_l$ is not second-countable. Suppose $\mathcal{B}$ is a countable basis for $\mathbb{R}_l$. For each $x \in \mathbb{R}$, there exists a open set $[x, +\infty)$ containing $x$. Thus, there exists a basis element $B_x \in \mathcal{B}$ such that $x \in B_x \subseteq [x, +\infty)$ which means $minB_x = x$. Since $\mathcal{B}$ is countable, the set of minimums $\{minB | B \in \mathcal{B}\}$ is also countable. This contradicts the uncountability of $\mathbb{R}$. Hence, $\mathbb{R}_l$ is not second-countable. However, $\mathbb{R}_l$ is first-countable since for each $x \in \mathbb{R}$, the collection of basis elements $\{[x, x + 1/n) | n \in \mathbb{N}\}$ forms a countable basis at $x$. 
\end{example}

\begin{property}
    $\mathbb{R}^{\omega}$ with box topology is not metrizable. (Hence $\mathbb{R}^J$ with box topology is not metrizable for infinite $J$. )
\end{property}
\par \noindent \textbf{Proof} 
\par We let 
\begin{equation}
    A = \{(x_1, x_2, \ldots) | x_i > 0, \forall i\}
\end{equation}
Then $0 \in \bar{A}$ in the box topology. 
\par We let 
\begin{equation}
    B = (a_1, b_1) \times (a_2, b_2) \times \cdots 
\end{equation}
such that $a_i < 0 < b_i$ for all $i$. Then $B$ is a basis element in the box topology containing $0$. And the intersection of $B$ and $A$ is not empty. 

We assert that there is no sequence in $A$ converging to $0$. Let $\{a_n \}_{n=1}^{\infty}$ be a sequence in $A$, where 
\begin{equation}
    a_n = (a_n^{(1)}, a_n^{(2)}, \ldots)
\end{equation}
with $a_n^{(i)} > 0$ for all $i, n$. We can construct a basis element $B'$ in the box topology containing $0$: 
\begin{equation}
    B' = (-a_1^{(1)}, a_1^{(1)}) \times (-a_2^{(2)}, a_2^{(2)}) \times \cdots
\end{equation}
Then $B'$ does not contain any $a_n$. Thus $\{a_n\}$ does not converge to $0$. 
\par So the sequence lemma tells us that $\mathbb{R}^{\omega}$ with box topology is not metrizable. 
\mbox{} \\ \null \hfill $\blacksquare$ 


\begin{property}
    Let $J$ be uncountable. Then $\mathbb{R}^{J}$ in product topology is not metrizable. 
\end{property}
\par \noindent \textbf{Proof} 
\par Take 
\begin{equation}
    A = \{(x_1, x_2, \cdots)| x_{\alpha} = 1 \text{ for all but finitely many coordinates}\}
\end{equation}
Then $0 \in \bar{A}$ in the product topology(Easy to verify). 
\par Given $n$, let $J_n$ denotes the subset of $J$ consisting of those indices $\alpha$ such that the $\alpha$-th coordinate of $a_n$ is different from $1$. Since $J_n$ is finite for each $n$, then the union $\bigcup_{n=1}^{\infty} J_n$ is countable. However, $J$ is uncountable, so there exists $\beta \in J$ such that $\beta \notin \bigcup_{n=1}^{\infty} J_n$. That is, for all $n$, the $\beta$-th coordinate of $a_n$ is equal to $1$. 
\par Let $\{a_n\}_{n=1}^{\infty}$ be a sequence in $A$, where there exists $\beta \in J$ such that $a_{\beta}^{(n)} = 1$ for all $n$. Then we can take 
\begin{equation}
    U = \pi_{\beta}^{-1}((- 1, 1))
\end{equation}
which is a basis element in the product topology containing $0$. However, $U$ does not contain any $a_n$. Thus $\{a_n\}$ does not converge to $0$.
\par By the sequence lemma, $\mathbb{R}^{J}$ in product topology is not metrizable.
\mbox{} \\ \null \hfill $\blacksquare$ 


\end{document}