% 声明为子文件,指定主文件
\documentclass[main.tex]{subfiles}

\begin{document}
\pagestyle{plain}
\setcounter{chapter}{13}

\chapter{Lecture14}
\label{chap:chapter14}

\begin{theorem}
    $\mathbb{R}^{\omega}$ is metrizable in the product topology.
\end{theorem}

\par \noindent \textbf{Proof} Define
\begin{equation}
    \bar{d}(x_i, y_i) = \min(|x_i - y_i|, 1)
\end{equation}
Let $x, y\in \mathbb{R}^{\omega}$. Then we define 
\begin{equation}
    D(x, y) = \sup \{ \frac{\bar{d}(x_i, y_i)}{i}| i\in \mathbb{Z}_{> 0}\}
\end{equation}
Then we can show that $D$ is a metric on $\mathbb{R}^{\omega}$ because we have 
\begin{equation}
    \bar{d}(x_i, z_i) \le \bar{d}(x_i, y_i) + \bar{d}(y_i, z_i) \Longrightarrow D(x, z) \le D(x, y) + D(y, z)
\end{equation}

\par Why does it define the product topology?
\par ($\Rightarrow$) Let $U$ be open in the metric topology. Then there exists $\epsilon >  0$ such that $B_D(x, \epsilon) \subseteq U$. Let $N$ be such that $\frac{1}{N} < \epsilon$. Then let 
\begin{equation}
    V = \prod_{i=1}^{N} (x_i - \epsilon, x_i + \epsilon) \times \prod_{i=N+1}^{\infty} \mathbb{R}
\end{equation}
Then let $y\in V$. We have
\begin{enumerate}
    \item If $i \le N$, then $\bar{d}(x_i, y_i) < \epsilon$.
    \item If $i > N$, then $\bar{d}(x_i, y_i) \le 1$.
\end{enumerate}
Thus, we have $D(x, y) = \sup \{\frac{\bar{d}(x_i, y_i)}{i}\} \le \max\{\epsilon, \frac{1}{N}\} = \epsilon$. So $y\in B_D(x, \epsilon) \subseteq U$. Hence, $V \subseteq U$. So $U$ is open in the product topology.
\par ($\Leftarrow$) Let $U$ be open in the product topology. Then 
\begin{equation}
    U = \prod U_i
\end{equation}
where $U_i$ is open if $i\in \{\alpha_1, \cdots, \alpha_n\}$ and $U_i = \mathbb{R}$ otherwise. We want $V$ open in the metric topology such that $V \subseteq U$. Let
\begin{equation}
    x \in U
\end{equation}
Then 
\begin{equation}
    x_i \in U_i
\end{equation}
for $i\in \{\alpha_1, \alpha_2, \cdots, \alpha_n\}$. Then there exists $\epsilon_i > 0$ such that
\begin{equation}
    (x_i - \epsilon_i, x_i + \epsilon_i) \subseteq U_i
\end{equation}
Let 
\begin{equation}
    \epsilon = \min (\{\frac{\epsilon_i}{i}| i \in \{\alpha_1, \alpha_2, \cdots, \alpha_n\}\}\cup \{1\})
\end{equation}
Then we claim that
\begin{equation}
    U_D(x, \epsilon) \subseteq U
\end{equation}
Let $y\in U_D(x, \epsilon)$. Then we have
\begin{equation}
    \frac{\bar{d}(x_i, y_i)}{i} \le D(x, y) < \epsilon
\end{equation}
\par If $i \in \{\alpha_1, \alpha_2, \cdots, \alpha_n\}$, then
\begin{equation}
    \frac{\bar{d}(x_i, y_i)}{i} < \epsilon \le \frac{\epsilon_i}{i}
\end{equation}
which implies that
\begin{equation}
    y_i \in (x_i - \epsilon_i, x_i + \epsilon_i) \subseteq U_i
\end{equation}
So $y\in U$. Hence, $U_D(x, \epsilon) \subseteq U$. So $U$ is open in the metric topology.
\mbox{} \\ \null \hfill $\blacksquare$ 

\begin{definition}
    $(X, \le )$ has a least upper bound property or supremum property if every non-empty subset of $X$ that is bounded above has a least upper bound in $X$. (For example, $\mathbb{R}$ has the least upper bound property)
\end{definition}

\par Let $X$ be a set with order topology and supremum property. 

\begin{theorem}
    $\forall a \le b \in X$, we have $[a, b]$ is compact. 
\end{theorem}
\par \noindent \textbf{Proof} Let $\mathcal{A}$ be an open covering of $[a, b]$. We need that there exists a finite subcovering of $\mathcal{A}$ covering $[a, b]$. 
\par \noindent \textbf{Step 1} Let $x\in [a, b]$ with $x \neq b$, there exists $y > x$ such that at most two elements of $\mathcal{A}$ can cover $[x, y]$. 
\begin{enumerate}
    \item If $x$ has an immediate successor $s(x)$, then $[x, s(x)] = \{x, s(x)\}$ is covered by two elements from $ \mathcal{A}$. 
    \item If not. Take $ x\in A$ with $A \in \mathcal{A}$ such that $\exists c\in X, x < c \le b, [x, c) \subseteq A$. Take any $y \in (x, c)$. Then $[x, y] \subseteq A$ is covered by one element from $\mathcal{A}$. 
\end{enumerate} 
\par \noindent \textbf{Step 2} Let $C$ be the set of points $y$ with $a\le y\le b$ such that $[a, y]$ has a finite subcovering from $\mathcal{A}$. Since $a \in C$, $C$ is non-empty. Let $c = \sup C$. 
\par \noindent \textbf{Step 3} Show that $c\in C$. 
\par We know from the first step that $c \neq a$. 
\par Choose $A\in \mathcal{A}$ such that $c\in A$. Then $\exists d \in [a, b]$ such that $(d, c] \subseteq A$. If $c\notin C$, then there must be some point $z \in (d, c)$ with $z \in C$. Then $[a, z]$ can be covered by finite elements(say $M$) from $\mathcal{A}$. Thus, $[a, c] \subseteq [a, z] \cup (d, c]$ can be covered by at most $M + 1$ elements from $\mathcal{A}$. So $c\in C$ which contradicts the assumption that $c \notin C$. Hence, $c\in C$. 
\par \noindent \textbf{Step 4} Show that $c = b$. If not, by Step 1, $\exists e > c$ such that $[c, e]$ can be covered by two elements from $\mathcal{A}$. Since $[a, c]$ can be covered by finite elements(say $M$) from $\mathcal{A}$, $[a, e]$ can be covered by at most $M + 2$ elements from $\mathcal{A}$. So $e\in C$ which contradicts the assumption that $c = \sup C$. Hence, $c = b$.  
\mbox{} \\ \null \hfill $\blacksquare$ 
\begin{corollary}
    $[a, b] \subseteq \mathbb{R}$ is compact. 
\end{corollary}

\begin{corollary}
    $[a_1, b_1] \times [a_2, b_2] \times \cdots \times [a_n, b_n] \subseteq \mathbb{R}^n$ is compact.
\end{corollary}

\begin{theorem}
    $A \subseteq \mathbb{R}^n$ is compact if and only if $A$ is closed and bounded.
\end{theorem}
\par \noindent \textbf{Proof} 
\par ($\Rightarrow$) Take a covering by 
\begin{equation}
    \bigcup U_N(0)
\end{equation} 
which is the union of open balls centered at $0$ with radius $N$ for $N \in \mathbb{Z}_{>0}$. Since $A$ is compact, there exists a finite subcovering. Thus, $A \subseteq U_{N_0}(0)$ for $N_0 = \max\{N_1, N_2, \cdots, N_k\}$. So $A$ is bounded. 
\par We know that $\mathbb{R}^n$ is a Hausdorff space. So $A$ is closed as a compact subset of a Hausdorff space.
\par ($\Leftarrow$) Since $A$ is bounded, there exists $n$ such that
\begin{equation}
    A \subseteq [a_1, b_1] \times [a_2, b_2] \times \cdots \times [a_n, b_n]
\end{equation}
We know that the closed subset of a compact set is compact. So $A$ is compact.
\mbox{} \\ \null \hfill $\blacksquare$ 

\begin{theorem}
    [Extreme value theorem]
    Let $f: X \rightarrow Y$ be continuous where $X$ is compact and $Y$ is an ordered set with the order topology. Then $\exists a, b \in X$ such that $\forall x\in X$, $f(a) \le f(x) \le f(b)$.
\end{theorem}

\par \noindent \textbf{Proof} 
\par We know that $A = f(X)$ is compact. We want to show that $A$ has a largest element. 
\par We assume the contrary that $A$ has no largest element. Then $\forall a \in A$, $\exists b \in A$ such that $b > a$. So 
\begin{equation}
    A = \bigcup_{a \in A} (-\infty, a)
\end{equation}
which is an open covering of $A$. Since $A$ is compact, there exists a finite subcovering. Let the largest element among the finite elements be $a_0$. Then $a_0$ is not covered. This is a contradiction. So $A$ has a largest element. Similarly, $A$ has a smallest element. 
\mbox{} \\ \null \hfill $\blacksquare$ 

\begin{definition}
    Let $(X, d)$ be a metric space, $A \subseteq X$, $x\in X$. Let $d(x, A) = \inf \{d(x, a)| a\in A\}$.
\end{definition}
\begin{property}
    $d(x, A)$ is a continuous function for $A$ fixed. 
\end{property}
\par \noindent \textbf{Proof} 
\par We know 
\begin{equation}
    d(x, A) \le d(x, a) \le d(x, y) + d(y, a)
\end{equation}
for any $a\in A$. 
Then 
\begin{equation}
    d(x, A) - d(x, y) \le \inf \{d(y, a)| a\in A\} = d(y, A)
\end{equation}
So 
\begin{equation}
    d(x, A) - d(y, A) \le d(x, y) 
\end{equation}
\mbox{} \\ \null \hfill $\blacksquare$ 

\end{document}