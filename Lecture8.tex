% 声明为子文件,指定主文件
\documentclass[main.tex]{subfiles}

\begin{document}
\pagestyle{plain}
\setcounter{chapter}{7}

\chapter{Lecture 8}
\label{chap:Lecture8}

\begin{theorem}
    Let $A$ be a subset of the topological space $X$. 
    \begin{enumerate}
        \item $x\in \bar{A} \Longleftrightarrow $ every open set $U \subseteq X$ with $x \in U$ satisfies $U \cap A \neq \emptyset$.
        \item For a basis $\mathcal{B}$ of $\mathcal{T}$, we have $x \in \bar{A} \Longleftrightarrow$ every basis element $B \in \mathcal{B}$ with $x \in B$ satisfies $B \cap A \neq \emptyset$. 
    \end{enumerate}
\end{theorem}

\par \noindent \textbf{Proof to 1} 
\par We transform each implication to its contrapositive, thereby obtaining the logically equivalent statement $(\lnot P) \Longleftrightarrow (\lnot Q)$. Written out, it is the follwoing statement. $x\notin \bar{A}$ if and only if there exists an open set $U$ containing $x$ that does not intersect $A$. 
\par If $x \notin \bar{A}$, then the set $U = X - \bar{A}$ is an open set containing $x$ that does not intersect $A$.
\par Conversely, if there exists an open set $U$ containing $x$ which does not intersect $A$, then $X-U$ is a closed set containing $A$. Then $\bar{A} \subseteq X - U$. Since $x \notin X - U$, $x\notin \bar{A}$. 
\mbox{} \\ \null \hfill $\blacksquare$ 

\par \noindent \textbf{Exercise} Finish the proof of 2. 

\begin{definition}
    For $A \subseteq X$, $x\in X$ is a limit point of $A$ if every open set $U \subseteq X$ containing $x$ contains a point of $A$ different from $x$ itself. We denote $A'$ as the set of all limit points of $A$.
\end{definition}

\begin{theorem}
    $\bar{A} = A \cup A'$
\end{theorem}
\par \noindent \textbf{Proof} 
\par We need to show that $\bar{A} \subseteq A \cup A'$ and $A \cup A' \subseteq \bar{A}$.
\par $(\supseteq)$ If $x\in A'$, then every neighborhood $U$ of $x$ intersects $A$. Then $x\in \bar{A}$ by the previous theorem. If $x\in A$, then obviously $x\in \bar{A}$. Thus $A \cup A' \subseteq \bar{A}$.
\par $(\subseteq)$ Let $x\in \bar{A}$. If $x\in A$, then $x\in A \cup A'$. If $x \in \bar{A}\setminus A$, then every neighborhood $U$ of $x$ intersects $A$ at $y$ such that $y \neq x$. Thus $x\in A'$. Therefore, $\bar{A} \subseteq A \cup A'$.

\mbox{} \\ \null \hfill $\blacksquare$ 

\begin{corollary}
    A set $A$ is closed if and only if $A' \subseteq A$.
\end{corollary}

\begin{definition}
    For a sequence $\{ x_n \}$ in $X$, we say that $a$ is a limit of the sequence if for every open set $U$ containing $a$, there exists $N \in \mathbb{N}$ such that for all $n \geq N$, $x_n \in U$. A sequence is convergent if it has a limit.
\end{definition}

\par \noindent \textbf{Remark} This is equivalent to the following: every open set $U$ containing $a$ contains almost all elements of the sequence except finitely many. And Notice that "all but finitely many" is not the same as "infinitely many".

\begin{example}
    Let $X$ be a topological space with the discrete topology. Then a sequence $\{ x_n \}$ converges to $a$ if and only if there exists $N \in \mathbb{N}$ such that for all $n \geq N$, $x_n = a$.
\end{example}

\begin{example}
    Let $X = \mathbb{Q}$ with the topology defined by the usual metric. Then the sequence: $3, 3.1, 3.14, 3.141, 3.1415, \ldots$ does not converge in $X$ since its limit $\pi$ is not in $\mathbb{Q}$.
\end{example}

\begin{example}
    Let $X = \mathbb{R}_l$ with the lower limit topology($[a, b)$ are basis open sets). Then we say that a sequence converges in $\mathbb{R}^l$ if and only if it converges to $a$ in $\mathbb{R}$ with an extra condition: $x_n \ge a, \forall n \ge N$ for some $N\in \mathbb{N}$.  
\end{example}

\par \noindent \textbf{Exercise} Show that if $x_n$ converges in $\mathcal{T}$, then $x_n$ converges in any coarser topology $\mathcal{T}' \subseteq \mathcal{T}$.

\begin{example}
    $X = \mathbb{N}$, the basis $U_m = \{0, 1, 2 \cdots, m\}$. $\mathcal{T} = \{U_m | m\in \mathbb{N}\}\cup \{ \mathbb{N}, \emptyset\}$. Then $x_n$ converges to $a$ if and only if $a$ is essential upper bound of $\{ x_n \}$. In this case, the limit is not unique. 
\end{example}

\begin{example}
    Finite complement topology on $\mathbb{R}$. 
    \par For example take $\mathcal{T} = \{ \mathbb{R}, \emptyset\}\cup \{ \mathbb{R}\setminus \{p_1, p_2, \cdots, p_m\}| p_1, p_2, \cdots, p_m \in \mathbb{R}, m \in \mathbb{N}\}$. 
    \begin{enumerate}
        \item If $x_n$ is eventually constant with value $a$, then $x_n$ converges to $a$. It can't not converge to $b \neq a$ because we can take the open set $\mathbb{R} \setminus \{a\}$ which contains $b$ but does not contain almost all elements of the sequence.
        \item If $x_n = (- 1)^n$, then $x_n$ does not converge. If $x_n = a$ for infinitely many $n$ and $x_n = b$ for infinitely many $n$ ($a \neq b$), then $x_n$ does not converge. 
        \item If $x_n = \dfrac{(-1)^n}{n}$, then $x_n$ converges to any real number. Because for any open set $U$, $U$ contains all but finitely many real numbers. So $U$ contains almost all elements of the sequence. 
        \item If $x_n$ assumes every value finitely many times, then $x_n$ converges to any real number.  
    \end{enumerate}
\end{example}

\end{document}