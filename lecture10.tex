% 声明为子文件,指定主文件
\documentclass[main.tex]{subfiles}

\begin{document}
\pagestyle{plain}
\setcounter{chapter}{9}

\chapter{Lecture10}
\label{chap:chapter10}

\section{Product topology on $X\times Y$} 
\begin{theorem}
    Let $f: A \rightarrow X\times Y$, let $f=(f_1,f_2)$ where $f_1: A \rightarrow X$ and $f_2: A \rightarrow Y$. Then $f$ is continuous if and only if both $f_1$ and $f_2$ are continuous. We denote $f(a) = (f_1(a), f_2(a))$. 
\end{theorem}
\par \noindent \textbf{Proof} 
\par Let $\pi_1: X\times Y \rightarrow X$ and $\pi_2: X\times Y \rightarrow Y$ be the projection maps defined by $\pi_1(x,y) = x$ and $\pi_2(x,y) = y$. Then $f_1 = \pi_1 \circ f$ and $f_2 = \pi_2 \circ f$. Since $\pi_1$ and $\pi_2$ are continuous, if $f$ is continuous, then both $f_1$ and $f_2$ are continuous. 
\par Conversely, suppose both $f_1$ and $f_2$ are continuous. Let $U, V$ be open. Then $f^{-1}(U\times V) = f_1^{-1}(U) \cap f_2^{-1}(V)$ for any open set $U$ in $X$ and $V$ in $Y$. Since $f_1$ and $f_2$ are continuous, $f_1^{-1}(U)$ and $f_2^{-1}(V)$ are open in $A$. Thus, $f^{-1}(U\times V)$ is open in $A$. 
\mbox{} \\ \null \hfill $\blacksquare$ 

\par There are two ways to introduce product topology. 
\begin{enumerate}
    \item Take $U_1 \subseteq X_1, U_2 \subseteq X_2, \cdots$. Then we can define the topology on $X_1 \times X_2 \times \cdots$ by the basis $\{U_1 \times U_2 \times \cdots \}$. This is called the \textbf{box topology}.
    \item Take $U_1 \subseteq X_1, U_2 \subseteq X_2, \cdots$ but only finitely many of them are not equal to the whole space. Then we can define the topology on $X_1 \times X_2 \times \cdots$ by the basis $\{U_1 \times U_2 \times \cdots \}$ where only finitely many $U_i$ are not equal to $X_i$. This is called the \textbf{product topology}. 
\end{enumerate}

\begin{definition}
    Let $J$ be an arbitrary set. A $J$-tuple of elements from $X$ is a function $x:J \rightarrow X$. So $\alpha \in J \mapsto x(\alpha) = x_\alpha \in X$. And sometimes we denote the $J$-tuple by $(x_\alpha)_{\alpha \in J}$. 
\end{definition}

\begin{definition}
    Let $(A_{\alpha})_{\alpha \in J}$ be an indexed family of sets. 
    \begin{equation}
        X = \bigcup_{\alpha \in J} A_{\alpha}
    \end{equation} 
    The Cartesian product of the family $(A_{\alpha})_{\alpha \in J}$ is denoted by 
    \begin{equation}
        \prod_{\alpha \in J} A_{\alpha}
    \end{equation}
    which is defined as the set of all J-tuples of elements in $X$ such that $x_{\alpha} \in A_{\alpha}$ for each $\alpha \in J$, that is, the set of all functions 
    \begin{equation}
        x: J \rightarrow X \quad \text{such that} \quad x(\alpha) \in A_{\alpha} \forall \alpha \in J
    \end{equation}
    When $A_{\alpha} = X$, we have $\prod_{\alpha \in J} A_{\alpha} = X^J$, the set of all functions from $J$ to $X$. 
\end{definition}
\begin{definition}
    Let $(X_{\alpha})_{\alpha \in J}$ be an indexed family of topological spaces. The box topology on $\prod_{\alpha \in J} X_{\alpha}$ is given by the basis 
    \begin{equation}
        \left\{ \prod_{\alpha \in J} U_{\alpha} : U_{\alpha} \text{ is open in } X_{\alpha} \forall \alpha \in J \right\}
    \end{equation}
    Then taking two basis elements $\prod_{\alpha \in J} U_{\alpha}$ and $\prod_{\alpha \in J} V_{\alpha}$, their intersection is 
    \begin{equation}
        \left( \prod_{\alpha \in J} U_{\alpha} \right) \cap \left( \prod_{\alpha \in J} V_{\alpha} \right) = \prod_{\alpha \in J} (U_{\alpha} \cap V_{\alpha})
    \end{equation}
    which is also a basis element. 
\end{definition}
\begin{definition}
    A collection $\mathcal{S}$ of subsets of topological space $X$ is a \textbf{subbasis} if 
    \begin{equation}
        \bigcup_{S \in \mathcal{S}} S = X
    \end{equation}
\end{definition}
\begin{property}
    Let $\mathcal{S}$ be a subbasis for a space $X$. Then the collection of all finite intersections of elements of $\mathcal{S}$ forms a basis for the topology on $X$. Then the topology generated by $\mathcal{S}$ is the collection of all unions of finite intersections of elements of $\mathcal{S}$. 
\end{property}

\par \noindent \textbf{Exercise} Prove the above property.  

\begin{definition}
    For a given $\beta \in J$, we denote $\pi_{\beta}: \prod_{\alpha \in J} X_{\alpha} \rightarrow X_{\beta}$ the projection map defined by 
    \begin{equation}
        x \mapsto x_{\beta}
    \end{equation}
    Then let $\mathcal{S}_{\beta} = \{ \pi_{\beta}^{-1}(U_{\beta}) : U_{\beta} \text{ is open in } X_{\beta} \}$. Then 
    \begin{equation}
        \mathcal{S} = \bigcup_{\beta \in J} \mathcal{S}_{\beta}
    \end{equation}
    is a subbasis(Proved yourself). The topology defined by this subbasis is called the \textbf{product topology}. The basis 
    $\mathcal{B}$ is given by finite intersections of elements of $\mathcal{S}$ and 
    \begin{equation}
        \mathcal{B} \ni B = \prod_{\alpha \in J} U_{\alpha} \quad \text{where } U_{\alpha} \text{ is open in } X_{\alpha} \text{ and } U_{\alpha} = X_{\alpha} \text{ for all but finitely many } \alpha 
    \end{equation}
    Notice that if $|J| < \infty$, then the box topology and the product topology are the same. And also notice that the product topology is coarser than the box topology. 
\end{definition}

\begin{theorem}
    Let $\mathcal{J}$ be a set of indices. Let $X_{\alpha}$ be a topological space for each $\alpha \in \mathcal{J}$. Let $A_{\alpha} \subseteq X_{\alpha}$ with subspace topology for each $\alpha \in \mathcal{J}$. Then the product/box topology on $\prod_{\alpha \in \mathcal{J}} A_{\alpha}$ is the subspace topology inherited from the product/ box topology on $\prod_{\alpha \in \mathcal{J}} X_{\alpha}$. 
    \par In other words, let $A_{\alpha}$ be a subspace of $X_{\alpha}$ for each $\alpha \in \mathcal{J}$. Then $\prod_{\alpha \in \mathcal{J}} A_{\alpha}$ is a subspace of $\prod_{\alpha \in \mathcal{J}} X_{\alpha}$ with the product/box topology.  
\end{theorem}

\begin{theorem}
    Let $\mathcal{J}$ be a set of indices. Let $X_{\alpha}$ be a Hausdorff topological space for each $\alpha \in \mathcal{J}$. Then the product/ box topology on $\prod_{\alpha \in \mathcal{J}} X_{\alpha}$ is Hausdorff.
\end{theorem}

\begin{theorem}
    Let $\mathcal{J}$ be a set of indices. Let $X_{\alpha}$ be a topological space for each $\alpha \in \mathcal{J}$. Let $A_{\alpha} \subseteq X_{\alpha}$ with subspace topology for each $\alpha \in \mathcal{J}$. Then $\prod_{\alpha \in \mathcal{J}} \bar{A}_{\alpha} = \overline{\prod_{\alpha \in \mathcal{J}} A_{\alpha}}$ with the product/ box topology.
\end{theorem}

\par \noindent \textbf{Proof}
\par We give a proof for the box topology case. The product topology case is similar. 
\par Take $(x_{\alpha})_{\alpha \in \mathcal{J}} = x \in \prod_{\alpha \in \mathcal{J}} \bar{A}_{\alpha}$. Let $\prod_{\alpha \in \mathcal{J}} U_{\alpha}$ be a basis open set containing $x$ where $U_{\alpha}$ is open in $X_{\alpha}$. Since $x_{\alpha} \in \bar{A}_{\alpha}$, there exists $y_{\alpha} \in U_{\alpha} \cap A_{\alpha}$ for each $\alpha \in \mathcal{J}$. Thus, $(y_{\alpha})_{\alpha \in \mathcal{J}} \in \prod_{\alpha \in \mathcal{J}} U_{\alpha} \cap \prod_{\alpha \in \mathcal{J}} A_{\alpha}$. Therefore, $x \in \overline{\prod_{\alpha \in \mathcal{J}} A_{\alpha}}$. So we have $\prod_{\alpha \in \mathcal{J}} \bar{A}_{\alpha} \subseteq \overline{\prod_{\alpha \in \mathcal{J}} A_{\alpha}}$. 
\par Conversely, take $x = (x_{\alpha})_{\alpha \in \mathcal{J}} \in \overline{\prod_{\alpha \in \mathcal{J}} A_{\alpha}}$. Let $\beta$ be an arbitrary index in $\mathcal{J}$. Take $V_{\beta}$ be an open set containing $x_{\beta}$. Then its preimage $\pi_{\beta}^{-1}(V_{\beta})$ is open in $\prod_{\alpha \in \mathcal{J}} X_{\alpha}$. Since $x\in \overline{\prod_{\alpha \in \mathcal{J}} A_{\alpha}}$, there exists $y = (y_{\alpha})_{\alpha \in \mathcal{J}} \in \pi_{\beta}^{-1}(V_{\beta}) \cap \prod_{\alpha \in \mathcal{J}} A_{\alpha}$. So $y_{\beta} \in V_{\beta} \cap A_{\beta}$. Thus, $x_{\beta} \in \bar{A}_{\beta}$. Since $\beta$ is arbitrary, we have $x \in \prod_{\alpha \in \mathcal{J}} \bar{A}_{\alpha}$. Therefore, $\overline{\prod_{\alpha \in \mathcal{J}} A_{\alpha}} \subseteq \prod_{\alpha \in \mathcal{J}} \bar{A}_{\alpha}$.

\par Two inclusions together give the desired equality. 
\mbox{} \\ \null \hfill $\blacksquare$ 

\begin{theorem}
    Let $f: A \rightarrow \prod_{\alpha \in \mathcal{J}} X_{\alpha}$ be given by $f(a) = (f_{\alpha}(a))_{\alpha \in \mathcal{J}}$ where $f_{\alpha}: A \rightarrow X_{\alpha}$ for each $\alpha \in \mathcal{J}$. Then $f$ is continuous if and only if each $f_{\alpha}$ is continuous. 
\end{theorem}

\par \noindent \textbf{Proof} 
\par Let $\pi_{\beta}$ be the projecction of the product onto its $\beta$-th factor. Then function $\pi_{\beta} $ is continuous because for any open set $U_{\beta}$ in $X_{\beta}$, we have $\pi_{\beta}^{-1}(U_{\beta})$ is a subbaisis element for the product topology. Since $f_{\beta} = \pi_{\beta} \circ f$, if $f$ is continuous, then each $f_{\beta}$ is continuous. 
\par Conversely, To prove $f$ is continuous, it suffices to show that the preimage of each subbasis element is open in $A$. Take any subbasis element $\pi_{\beta}^{-1}(U_{\beta})$ where $U_{\beta}$ is open in $X_{\beta}$. Then
\begin{equation}
    f^{-1}(\pi_{\beta}^{-1}(U_{\beta})) = f_{\beta}^{-1}(U_{\beta})
\end{equation}
Since $f_{\beta}$ is continuous, $f_{\beta}^{-1}(U_{\beta})$ is open in $A$. Thus, $f$ is continuous.
\mbox{} \\ \null \hfill $\blacksquare$ 

\begin{example}
    For box topology, the above theorem may fail. 
    \par Consider $\mathbb{R}^{\omega}$, the countably infinite product of $\mathbb{R}$ with itself. Let us define $f: \mathbb{R} \rightarrow \mathbb{R}^{\omega}$ by
    \begin{equation}
        f(t) = (t, t, t, \ldots)
    \end{equation}
    Then each coordinate function $f_n: \mathbb{R} \rightarrow \mathbb{R}$ defined by $f_n(t) = t$ is continuous; therefore the function $f$ is continuous given the product topology on $\mathbb{R}^{\omega}$. However, $f$ is not continuous when $\mathbb{R}^{\omega}$ is given the box topology. Consider, the basis element
    \begin{equation}
        B = \prod_{n=1}^{\infty} \left( -\frac{1}{n}, \frac{1}{n} \right) 
    \end{equation}
    for the box topology on $\mathbb{R}^{\omega}$. 
    \par We assert that $f^{-1}(B)$ is not open in $\mathbb{R}$. If it were open, then there would exist $\epsilon > 0$ such that $(-\epsilon, \epsilon) \subseteq f^{-1}(B)$. This would mean $f((-\epsilon, \epsilon)) \subseteq B$, so that 
    \begin{equation}
        f_n((-\epsilon, \epsilon)) = (-\epsilon, \epsilon) \subseteq \left( -\frac{1}{n}, \frac{1}{n} \right)
    \end{equation}
    for each $n$. But this is impossible for $n$ sufficiently large that $\frac{1}{n} < \epsilon$. Thus, $f^{-1}(B)$ is not open, and $f$ is not continuous.
\end{example}


\end{document}