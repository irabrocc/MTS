% 声明为子文件,指定主文件
\documentclass[main.tex]{subfiles}

\begin{document}
\pagestyle{plain}
\setcounter{chapter}{9}

\chapter{Lecture10}
\label{chap:chapter10}

\section{Product topology on $X\times Y$} 
\begin{theorem}
    Let $f: A \rightarrow X\times Y$, let $f=(f_1,f_2)$ where $f_1: A \rightarrow X$ and $f_2: A \rightarrow Y$. Then $f$ is continuous if and only if both $f_1$ and $f_2$ are continuous. We denote $f(a) = (f_1(a), f_2(a))$. 
\end{theorem}
\par \noindent \textbf{Proof} 
\par Let $\pi_1: X\times Y \rightarrow X$ and $\pi_2: X\times Y \rightarrow Y$ be the projection maps defined by $\pi_1(x,y) = x$ and $\pi_2(x,y) = y$. Then $f_1 = \pi_1 \circ f$ and $f_2 = \pi_2 \circ f$. Since $\pi_1$ and $\pi_2$ are continuous, if $f$ is continuous, then both $f_1$ and $f_2$ are continuous. 
\par Conversely, suppose both $f_1$ and $f_2$ are continuous. Let $U, V$ be open. Then $f^{-1}(U\times V) = f_1^{-1}(U) \cap f_2^{-1}(V)$ for any open set $U$ in $X$ and $V$ in $Y$. Since $f_1$ and $f_2$ are continuous, $f_1^{-1}(U)$ and $f_2^{-1}(V)$ are open in $A$. Thus, $f^{-1}(U\times V)$ is open in $A$. 
\mbox{} \\ \null \hfill $\blacksquare$ 

\par There are two ways to introduce product topology. 
\begin{enumerate}
    \item Take $U_1 \subseteq X_1, U_2 \subseteq X_2, \cdots$. Then we can define the topology on $X_1 \times X_2 \times \cdots$ by the basis $\{U_1 \times U_2 \times \cdots \}$. This is called the \textbf{box topology}.
    \item Take $U_1 \subseteq X_1, U_2 \subseteq X_2, \cdots$ but only finitely many of them are not equal to the whole space. Then we can define the topology on $X_1 \times X_2 \times \cdots$ by the basis $\{U_1 \times U_2 \times \cdots \}$ where only finitely many $U_i$ are not equal to $X_i$. This is called the \textbf{product topology}. 
\end{enumerate}

\begin{definition}
    Let $J$ be an arbitrary set. A $J$-tuple of elements from $X$ is a function $x:J \rightarrow X$. So $\alpha \in J \mapsto x(\alpha) = x_\alpha \in X$. And sometimes we denote the $J$-tuple by $(x_\alpha)_{\alpha \in J}$. 
\end{definition}

\begin{definition}
    Let $(A_{\alpha})_{\alpha \in J}$ be an indexed family of sets. 
    \begin{equation}
        X = \bigcup_{\alpha \in J} A_{\alpha}
    \end{equation} 
    The Cartesian product of the family $(A_{\alpha})_{\alpha \in J}$ is denoted by 
    \begin{equation}
        \prod_{\alpha \in J} A_{\alpha}
    \end{equation}
    which is defined as the set of all J-tuples of elements in $X$ such that $x_{\alpha} \in A_{\alpha}$ for each $\alpha \in J$, that is, the set of all functions 
    \begin{equation}
        x: J \rightarrow X \quad \text{such that} \quad x(\alpha) \in A_{\alpha} \forall \alpha \in J
    \end{equation}
    When $A_{\alpha} = X$, we have $\prod_{\alpha \in J} A_{\alpha} = X^J$, the set of all functions from $J$ to $X$. 
\end{definition}
\begin{definition}
    Let $(X_{\alpha})_{\alpha \in J}$ be an indexed family of topological spaces. The box topology on $\prod_{\alpha \in J} X_{\alpha}$ is given by the basis 
    \begin{equation}
        \left\{ \prod_{\alpha \in J} U_{\alpha} : U_{\alpha} \text{ is open in } X_{\alpha} \forall \alpha \in J \right\}
    \end{equation}
    Then taking two basis elements $\prod_{\alpha \in J} U_{\alpha}$ and $\prod_{\alpha \in J} V_{\alpha}$, their intersection is 
    \begin{equation}
        \left( \prod_{\alpha \in J} U_{\alpha} \right) \cap \left( \prod_{\alpha \in J} V_{\alpha} \right) = \prod_{\alpha \in J} (U_{\alpha} \cap V_{\alpha})
    \end{equation}
    which is also a basis element. 
\end{definition}
\begin{definition}
    A collection $\mathcal{S}$ of subsets of topological space $X$ is a \textbf{subbasis} if 
    \begin{equation}
        \bigcup_{S \in \mathcal{S}} S = X
    \end{equation}
\end{definition}
\begin{property}
    Let $\mathcal{S}$ be a subbasis for a space $X$. Then the collection of all finite intersections of elements of $\mathcal{S}$ forms a basis for the topology on $X$. Then the topology generated by $\mathcal{S}$ is the collection of all unions of finite intersections of elements of $\mathcal{S}$. 
\end{property}

\par \noindent \textbf{Exercise} Prove the above property.  

\begin{definition}
    For a given $\beta \in J$, we denote $\pi_{\beta}: \prod_{\alpha \in J} X_{\alpha} \rightarrow X_{\beta}$ the projection map defined by 
    \begin{equation}
        x \mapsto x_{\beta}
    \end{equation}
    Then let $\mathcal{S}_{\beta} = \{ \pi_{\beta}^{-1}(U_{\beta}) : U_{\beta} \text{ is open in } X_{\beta} \}$. Then 
    \begin{equation}
        \mathcal{S} = \bigcup_{\beta \in J} \mathcal{S}_{\beta}
    \end{equation}
    is a subbasis(Proved yourself). The topology defined by this subbasis is called the \textbf{product topology}. The basis 
    $\mathcal{B}$ is given by finite intersections of elements of $\mathcal{S}$ and 
    \begin{equation}
        \mathcal{B} \ni B = \prod_{\alpha \in J} U_{\alpha} \quad \text{where } U_{\alpha} \text{ is open in } X_{\alpha} \text{ and } U_{\alpha} = X_{\alpha} \text{ for all but finitely many } \alpha 
    \end{equation}
    Notice that if $|J| < \infty$, then the box topology and the product topology are the same. And also notice that the product topology is coarser than the box topology. 
\end{definition}

\begin{theorem}
    Let $A_{\alpha} \subseteq X_{\alpha}$ 
\end{theorem}

\end{document}