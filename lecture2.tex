% 声明为子文件,指定主文件
\documentclass[main.tex]{subfiles}

\begin{document}
\pagestyle{plain}
\setcounter{chapter}{1}

\chapter{Later}
\label{chap:chapter1}

\begin{definition}
    For $p>1, q>1$. $p$ is conjugate to $q$ if $\frac{1}{p} + \frac{1}{q} = 1$.
\end{definition}
\par \noindent \textbf{Remark} When $p = 1$, $q = \infty$; when $p = \infty$, $q = 1$. And $2$ is conjugate to itself. 

\begin{lemma}
    [Jensen's inequality]
    If $f(t)$ is a strictly concave function on an interval $I \subseteq \mathbb{R}$, then for any $t_1, t_2\in I$ and any $\lambda_1, \lambda_2 > 0$ with $\lambda_1 + \lambda_2 = 1$, we have
    \begin{equation}
        f(\lambda_1 t_1 + \lambda_2 t_2) \ge \lambda_1 f(t_1) + \lambda_2 f(t_2)
    \end{equation}
    where the equality holds if and only if $t_1 = t_2$.
\end{lemma}
\par \noindent \textbf{Proof} Obviuos. 

\begin{theorem}
    [Young's inequality]
    For $a, b \ge 0$ and $p, q > 1$ with $\frac{1}{p} + \frac{1}{q} = 1$, we have
    \begin{equation}
        ab \le \dfrac{a^p}{p} + \frac{b^q}{q}
    \end{equation}
\end{theorem}
\par \noindent \textbf{Proof} 
\par Let $x = a^p, y = b^q$. Then $a = x^{\frac{1}{p}}, b = y^{\frac{1}{q}}$. Substituting these into the inequality, we need to show that
\begin{equation}
    x^{\frac{1}{p}} y^{\frac{1}{q}} \le \dfrac{x}{p} + \dfrac{y}{q}, \forall x, y \ge 0
\end{equation}
If $x = 0$ or $y = 0$, the inequality holds trivially. Now we assume $x, y > 0$. Let $f(t) = \ln t$, which is strictly concave on $(0, +\infty)$. By Jensen's inequality, we have
\begin{equation}
    f\left( \dfrac{x}{p} + \dfrac{y}{q} \right) \ge \dfrac{1}{p} f(x) + \dfrac{1}{q} f(y) 
\end{equation}
Exponentiating both sides, we have
\begin{equation}
    \dfrac{x}{p} + \dfrac{y}{q} \ge x^{\frac{1}{p}} y^{\frac{1}{q}}
\end{equation}
\mbox{} \\ \null \hfill $\blacksquare$ 

\begin{theorem}
    [Hölder's inequality]
    For $p, q \ge 1$ with $\frac{1}{p} + \frac{1}{q} = 1$, for any $a = (a_i)\in l_p, b = (b_i) \in l_q$ respectively, we have \[ \sum_{k=1}^{\infty} |a_k b_k| \le \sqrt[\frac{1}{p}]{\sum a_k^p} \sqrt[\frac{1}{q}]{\sum b_k^q}  \]
\end{theorem}
\par \noindent \textbf{Proof} 
\par In case when $\|a\|_p = 0$ or $\|b\|_q = 0$, the inequality holds trivially. Now we assume $\|a\|_p > 0$ and $\|b\|_q > 0$. Let 
\begin{equation}
    x_k = \frac{|a_k|}{\|a\|_p}, \quad y_k = \frac{|b_k|}{\|b\|_q}, \quad \forall k \ge 1
\end{equation}
We have 
\begin{equation}
    \sum_{k=1}^{\infty} x_k^p = \frac{|a_k|^p}{\|a\|_p^p} = 1, \quad \sum_{k=1}^{\infty} y_k^q = \frac{|b_k|^q}{\|b\|_q^q} = 1
\end{equation}
\par By Young's inequality, we have
\begin{equation}
    x_k y_k \le \dfrac{x_k^p}{p} + \dfrac{y_k^q}{q}, \quad \forall k \ge 1
\end{equation}
\par Summing over $k$ from $1$ to $\infty$, we have
\begin{equation}
    \sum_{k=1}^{\infty} x_k y_k \le \dfrac{1}{p} \sum_{k=1}^{\infty} x_k^p + \dfrac{1}{q} \sum_{k=1}^{\infty} y_k^q = \dfrac{1}{p} + \dfrac{1}{q} = 1
\end{equation}
\par Thus, we have
\begin{equation}
    \sum_{k=1}^{\infty} |a_k b_k| = \|a\|_p \|b\|_q \sum_{k=1}^{\infty} x_k y_k \le \|a\|_p \|b\|_q
\end{equation}
\par For the case when $p = 1$ and $q = \infty$, we have
\begin{equation}
    \sum_{k=1}^{\infty} |a_k b_k| \le \sum_{k=1}^{\infty} |a_k| \|b\|_{\infty} = \|a\|_1 \|b\|_{\infty}
\end{equation}
\\ \null \hfill $\blacksquare$

\begin{theorem}
    [Minkowski inequality]
    For $p \ge 1$, for any $x, y \in l_p$, we have \[ \|x + y\|_p \le \|x\|_p + \|y\|_p. \]
\end{theorem}
\par \noindent \textbf{Proof}  
\par If $\sum_{k = 1}^{\infty} |x_k + y_k|^p = 0$, then $x_k + y_k = 0$ for all $k$. Thus, $\|x\|_p = \|y\|_p = 0$ and the inequality holds trivially. Now we assume $\sum_{k = 1}^{\infty} |x_k + y_k|^p > 0$. 
\par Let us first show that for $x = (x_i), y = (y_i) \in l_p$, we have \[ \sqrt[\frac{1}{p}]{|x_k + y_k|^p} \le \sqrt[\frac{1}{p}]{|x_k|^p} + \sqrt[\frac{1}{p}]{|y_k|^p} \] 
\par For every summand, we have 
\begin{align*}
    |x_k + y_k|^p & = |x_k + y_k| \cdot |x_k + y_k|^{p-1} \\
    & \le |x_k| \cdot |x_k + y_k|^{p-1} + |y_k| \cdot |x_k + y_k|^{p-1} \\
\end{align*}
So we have 
\begin{equation}
    \sum_{k=1}^{n} |x_k + y_k|^p \le \sum_{k=1}^{n} |x_k| \cdot |x_k + y_k|^{p-1} + \sum_{k=1}^{n} |y_k| \cdot |x_k + y_k|^{p-1}
\end{equation}
\par Let $a_k = |x_k|, b_k = |x_k + y_k|^{p-1}$, then $b_k^q = |x_k + y_k|^p$. By Hölder's inequality, we have 
\[
    \sum_{k=1}^{n} |x_k| \cdot |x_k + y_k|^{p-1} \le \sqrt[\frac{1}{p}]{\sum_{k=1}^{n} |x_k|^p} \sqrt[\frac{1}{q}]{\sum_{k=1}^{n} |x_k + y_k|^p}
\]
On the other hand, let $a_k = |y_k|, b_k = |x_k + y_k|^{p-1}$, we have
\[
    \sum_{k=1}^{n} |y_k| \cdot |x_k + y_k|^{p-1} \le \sqrt[\frac{1}{p}]{\sum_{k=1}^{n} |y_k|^p} \sqrt[\frac{1}{q}]{\sum_{k=1}^{n} |x_k + y_k|^p}
\]
Combining these two inequalities, we have
\begin{equation}
    \sum_{k=1}^{n} |x_k + y_k|^p \le \left( \sqrt[\frac{1}{p}]{\sum_{k=1}^{n} |x_k|^p} + \sqrt[\frac{1}{p}]{\sum_{k=1}^{n} |y_k|^p} \right) \sqrt[\frac{1}{q}]{\sum_{k=1}^{n} |x_k + y_k|^p}
\end{equation}
\par Let $S_n = \sum_{k = 1}^{n} |x_k + y_k|^p$, then $
    S_n \le C S_n^{\frac{1}{q}}$
If $S_n = 0$, then $S_n \le C^p$. If $S_n > 0$, then $S_n^{1 - \frac{1}{q}} = S_n^{\frac{1}{p}} \le C$ which implies $S_n \le C^p$. So $S_n$ is increasing and bounded above. Thus, $\lim_{n \to \infty} S_n = \sum_{k=1}^{\infty} |x_k + y_k|^p$ exists and is finite.
\par Dividing both sides by $\sqrt[\frac{1}{q}]{\sum_{k=1}^{n} |x_k + y_k|^p}$, we have
\begin{equation}
    \sqrt[\frac{1}{p}]{\sum_{k=1}^{n} |x_k + y_k|^p} \le \sqrt[\frac{1}{p}]{\sum_{k=1}^{n} |x_k|^p} + \sqrt[\frac{1}{p}]{\sum_{k=1}^{n} |y_k|^p}
\end{equation}
\par It's easy to show that each term on the right-hand side converges as $n \to \infty$ because $x, y \in l_p$ and we know that a monotonically increasing sequence bounded above converges. So we have that the both sides converge as $n \to \infty$. 
\par Taking limit $n \to \infty$, by continuity of $n$-th root, we have the desired result.
\\ \null \hfill $\blacksquare$ 

\par Let $X$ be a metric space. Let $x\in X$. We have the following definitions.
\begin{definition}
    We define a neighborhood of $x$ to be a set of the form \[ U_{\epsilon} (x) = \{y \in X | d(x, y) < \epsilon \} \] for some $\epsilon > 0$.
\end{definition}
\begin{definition}
    We define a punctured neighborhood of $x$ to be a set of the form \[ U_{\epsilon}^{\ast} (x) = \{y \in X | 0 < d(x, y) < \epsilon \} = U_{\epsilon} (x) \setminus \{x\} \] for some $\epsilon > 0$. 
\end{definition}

\begin{definition}
    We say that $M \subseteq X$ is open in $X$ if for every $x \in M$, there exists $\epsilon > 0$ such that $U_{\epsilon} (x) \subseteq M$.
\end{definition}
\par \noindent \textbf{Remark} $\emptyset, X$ are open in $X$ by definition. 

\begin{example}
    Is it possible that in a metric space $X$, a ball is contained properly inside a ball with smaller radius? That is, is there $x \in X$ and $0 < r < s$ such that $U_s (x) \subsetneq U_r (x)$?
    [Hint: If $Y \subseteq X$ and $(X, d)$ is a metric space, then $(Y, d)$ is also a metric space. ]
\end{example}

\par \noindent \textbf{Solution} Yes. Let $X = (-1, 1)$ with the usual metric. Then $U_{\frac{3}{2}} (\dfrac{4}{3}) = (-\frac{4}{3}, 1) \subsetneq U_1 (0) = (-1, 1)$. 

\begin{example}
    Draw balls centered at $0$ in $\mathbb{R}^2$ with norms $\|\cdot\|_1, \|\cdot\|_2, \|\cdot\|_{\infty}$. 
\end{example}

\begin{example} 
    "Amazon Metric" on $\mathbb{R}^2$ is given by 
    \[d((x_1, y_1), (x_2, y_2)) = \begin{cases}
    |y_1 - y_2|   &  \text{if } x_1 = x_2  \\
    |y_1| + |x_1 - x_2| + |y_2|   &  \text{if } x_1 \neq x_2
   \end{cases}\] 
\end{example}

\par We will solve these problems later in the course.

\begin{theorem} 
    \begin{enumerate}
        \item The intersection of finitely many open sets is open, that is, $U_1 \cap U_2 \cap \cdots \cap U_n$ is open where each $U_i$ is open in $X$. 
        \item The union of any collection of open sets is open, that is, if $\{U_i\}_{i \in I}$ is a collection of open sets in $X$, then $\bigcup_{i \in I} U_i$ is open.
    \end{enumerate}
\end{theorem}

\par \noindent \textbf{Proof to 1.} 
\par If $V = U_1 \cap U_2 \cap \cdots \cap U_n = \emptyset$, then $V$ is open by definition.
\par If $V \neq \emptyset$, let $x \in V$. Since $x \in U_i$ for each $i = 1, 2, \ldots, n$, there exists $\epsilon_i > 0$ such that $U_{\epsilon_i} (x) \subseteq U_i$. Let $\epsilon = \min\{\epsilon_1, \epsilon_2, \ldots, \epsilon_n\}$. Then $U_{\epsilon} (x) \subseteq U_i$ for each $i$, so $U_{\epsilon} (x) \subseteq V$. Thus, $V$ is open.
\par \noindent \textbf{Proof to 2.}
\par Let $x \in U = \bigcup_{i \in I} U_i$. Then there exists some $j \in I$ such that $x \in U_j$. Since $U_j$ is open, there exists $\epsilon > 0$ such that $U_{\epsilon} (x) \subseteq U_j \subseteq U$. Thus, $U$ is open. 

\end{document}