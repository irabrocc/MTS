% 声明为子文件,指定主文件
\documentclass[main.tex]{subfiles}

\begin{document}
\pagestyle{plain}
\setcounter{chapter}{4}

\chapter{Later}
\label{chap:chapter5}

\begin{theorem}
    The following are equivalent: 
    \begin{enumerate}
        \item The Intermediate Value Theorem holds for $X$: for any continuous map $f: X \rightarrow \mathbb{R}$, and any $c$ between $f(x_1)$ and $f(x_2)$ for some $x_1, x_2 \in X$, there exists some $x \in X$ such that $f(x) = c$.
        \item There is no continuous surjection from $X$ onto a discrete two-point space $\{0,1\}$.
        \item $X$ cannot be partitioned into two nonempty disjoint open sets.
        \item $X$ cannot be partitioned into two nonempty disjoint closed sets.
        \item Every proper subset of $X$ that is both open and closed is either $\emptyset$ or $X$ itself. 
    \end{enumerate}
\end{theorem}

\par \noindent \textbf{Remark} If $X$ is path-connected, then it is connected. The converse is not necessarily true.
\begin{lemma}
    The closure of a connected set is connected.
\end{lemma}
\par \noindent \textbf{Proof} Let $A$ be a connected set and $\bar{A}$ be its closure. Suppose $\bar{A} = U_1 \cup U_2$ where $U_1$ and $U_2$ are nonempty disjoint open sets in $\bar{A}$. Then $A = (A \cap U_1) \cup (A \cap U_2)$ is non-trivial: let $U_1\cap A = \emptyset$, then $U_1 \subseteq \bar{A} \setminus A$ which is impossible. 

\begin{example}
    Let $X = \{(x, y)| y = \sin(\frac{1}{x}), x > 0\} \cup  \{(0, y)| y \in [-1, 1]\} \subset \mathbb{R}^2$ with the Euclidean metric. Then $X$ is connected but not path-connected.
\end{example}
\par \noindent \textbf{Proof} 
\par First we prove that $X$ is not path-connected. Let $p_1, p_2 \in X$ such that $P_1 = (0, 0)$ and $P_2 = (1, \sin(1))$. Let $\gamma: [0,1] \rightarrow X$ with $\gamma(t) = (x(t), y(t))$ such that $\gamma(0) = P_1$ and $\gamma(1) = P_2$. Since $\gamma$ is continuous, both $x(t)$ and $y(t)$ are continuous. 
\par Let $U = \{\tau| x(t) = 0\} \subseteq [0,1]$. Since $U$ is a preimage of a closed set under a continuous map, $U$ is closed. Thus, $t_0 = \sup U$ exists. Note that $t_0 < 1$ since $x(1) = 1$.
\par Let $\varPhi(t) = \dfrac{1}{x(t)}$ which is well-defined on for $t > t_0$. Notice that $\varPhi(t) \rightarrow +\infty$ as $t \rightarrow t_0^+$. Then, for $t > t_0$, $y(t) = \sin(\varPhi(t))$. Take the interval $(t_0, t_0 + \epsilon)$, then $y(t)$ oscillates between $-1$ and $1$ infinitely many times. So $y(t)$ is not continuous at $t_0$. This is a contradiction. Thus, no such path $\gamma$ exists and $X$ is not path-connected.


\par Second, we prove that $X$ is connected. Let $X = Y \cup \{(0, y)| y\in [-1, 1]\}$. We know that $\bar{Y} = X$ and $Y$ is path-connected (thus connected). Then using the property that the closure of a connected set is connected, we have $X$ is connected.

\par \noindent \textbf{Remark} Also, we can prove it using 2 in the \ref{thm:5.0.1}. 

\subsection{Connected Components}
\par For any $x, y\in X$, we say that $x \sim y$ if $\exists \gamma : [0,1] \rightarrow X$ continuous such that $\gamma(0) = x$ and $\gamma(1) = y$. It is easy to check that $\sim$ is an equivalence relation. We call the equivalence classes the connected components of $X$.
\par \noindent \textbf{Exercise} Define connected components for the notion of connected (not just path-connected) spaces. 

\par \noindent \textbf{Observation} The number of connected components is preserved under homeomorphisms. 
\begin{example}
    $(0, 1) \cup (2,3)$ is not homeomorphic to $(0,1)$ since the former has 2 connected components while the latter has 1 connected component. 
\end{example}

\begin{example}
    The interval $[0, 1]$ is not homeomorphic to $[0, 1]\times [0, 1]$. Removing a point(not in the boundary) from $[0, 1]$ results in a disconneted space, while removing a point from $[0, 1]\times [0, 1]$ still results in a connected space. Thus, they are not homeomorphic. 
\end{example}

\begin{example}
    $(0, 1) \nsim [0, 1] \nsim [0, 1)$. Removing a boundary point from $[0, 1]$ still results in a connected space, while removing a boundary point from $(0, 1)$ results in a disconnected space. Thus, they are not homeomorphic.
\end{example}

\begin{example}
    We can split the alphabet(capital letter) into nine homeomorphism classes:
    \begin{itemize}
        \item A   
        \item B 
        \item C, I, J, L, M, N, S, U, V, W, Z
        \item D, O 
        \item E, F, G, T, Y 
        \item H 
        \item K, X, Z 
        \item P 
        \item Q, R 
    \end{itemize}
\end{example}

\begin{example}
    [First homology group] 
    $$2026 \leadsto 2$$
    $$2025 \leadsto 1$$
    $$1949 \leadsto 2$$
    $$1982 \leadsto 3$$
    $$1988 \leadsto 5$$
\end{example}

\begin{property}
    $U \subseteq \mathbb{R}^2$ open in one of the metrics defined by the norms listed below iff it is open in the others:
    \begin{itemize}
        \item Euclidean norm: $||(x,y)||_2 = \sqrt{x^2 + y^2}$
        \item Supremum norm: $||(x,y)||_\infty = \max{|x|, |y|}$
        \item Diamond norm: $||(x,y)||_{1/2} = \sqrt{|x|} + \sqrt{|y|}$
    \end{itemize}
    In other words, these metrics define the same topology. So they have the same open sets. 
\end{property}

\begin{definition}
    Two norms on $V$ $||\cdot||_a$ and $||\cdot||_b$ are called equivalent if $\exists c_1, c_2$ such that:
    \begin{equation}
        c_1 ||v||_a \leq ||v||_b \leq c_2 ||v||_a
    \end{equation}
\end{definition}

\par \noindent \textbf{Exercise} 
\begin{enumerate}
    \item This is an equivalent relation.
    \item Equivalent norms define the same topology. 
    \item $||\cdot||_1$, $||\cdot||_2$, and $||\cdot||_\infty$ on $\mathbb{R}^n$ are equivalent.
    \item [*] All norms on $\mathbb{R}^n $ are equivalent. 
\end{enumerate}

\par The notes below is not for this lecture. 

\begin{theorem}
    All norms on $\mathbb{R}^n$ are equivalent. 
\end{theorem}
\par \noindent \textbf{Proof} 
\par Given $\|\cdot\|_a, \|\cdot\|_b$ two norms on $\mathbb{R}^n$. We define a function $f: \mathbb{R}^n \setminus \{0\} \rightarrow \mathbb{R}$ by $f(x) = \dfrac{||x||_a}{||x||_b}$ which is a continuous function. And $f(x) = f(\lambda x)$ for any $\lambda > 0$. So $f$ is completely determined by its values on the unit sphere $S^{n-1} = \{x \in \mathbb{R}^n | ||x||_a = 1\}$. Since $S^{n-1}$ is compact, $f$ attains its maximum and minimum on $S^{n-1}$, denoted as $m$ and $M$. Thus, for any $x \neq 0$, we have:
\begin{equation}
    0 < m \leq \dfrac{||x||_a}{||x||_b} \leq M < +\infty
\end{equation}
\par Therefore, $\|x\|_b$ and $\|x\|_a$ are equivalent. 

\par \noindent \textbf{Remark} In infinite-dimensional vector spaces, $S^{n-1}$ is not compact(However, it is closed and bounded). 

\par \noindent \textbf{Exercise} Give an example of norms on $l_1$(convergent series, i.e., $l_1 = \{(a_n) | \sum_{n=1}^{\infty} |a_n| < +\infty\}$).
\par \noindent \textbf{Solution} The trivial one is $||x|| = \sum_{n=1}^{\infty} |a_n|$. It is easy to verify as it is convergent. 



\end{document}