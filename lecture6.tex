% 声明为子文件,指定主文件
\documentclass[main.tex]{subfiles}

\begin{document}
\pagestyle{plain}
\setcounter{chapter}{5}

\chapter{Lecture5}
\label{chap:chapter6}

\section{Topological Spaces in general}

\begin{definition}
    A \textbf{topological space} is a pair $(X, \mathcal{U})$ where $X$ is a set and $\mathcal{U} \subseteq 2^X$ such that:
   \begin{enumerate}
        \item $\emptyset, X \in \mathcal{U}$. 
        \item If $U_{\alpha} \in \mathcal{U}$ for all $\alpha \in A$, then $\bigcup_{\alpha \in A} U_{\alpha} \in \mathcal{U}$.
        \item If $U_1, U_2, \ldots, U_n \in \mathcal{U}$, then $\bigcap_{i=1}^n U_i \in \mathcal{U}$.
   \end{enumerate} 
   \par $\mathcal{U}$ is called a \textbf{topology} on $X$. The elements of $\mathcal{U}$ are called \textbf{open sets}.
\end{definition}

\begin{example}
    Any metric space $(X, d)$ is a topological space, with $\mathcal{U}$ defined as: 
    \begin{equation}
        U\in \mathcal{U} \iff \forall x \in U, \exists r > 0 \text{ such that } B(x, r) \subseteq U.
    \end{equation}
    Open sets defined in this topology are the same as open sets defined by the metric. 
\end{example}
\begin{example}
    Let $X$ be any non-empty set. The \textbf{discrete topology} on $X$ is defined as $\mathcal{U} = 2^X$. Every subset of $X$ is open. 
\end{example}
\begin{example}\label{Not metric1}
    Let $X$ be any non-empty set. The \textbf{anti-discrete topology} on $X$ is defined as $\mathcal{U} = \{\emptyset, X\}$. Only $\emptyset$ and $X$ are open. \textbf{This example is not defined by a metric. }
\end{example}
\begin{example}
    [Finite complement topology]\label{Not metric2}
    Let $X$ be an infinite set, say, $\mathbb{R}$. Define  
    \begin{equation}
        U \in \mathcal{U} \Longleftrightarrow X \setminus U \text{ is finite or } U = \emptyset. 
    \end{equation}
\end{example}
\begin{property}
    Let $(X, d)$ be a metric space. Let $x, y\in X$, then there exists open sets $U, V$ such that $x \in U$, $y \in V$ and $U \cap V = \emptyset$. 
\end{property}

\begin{definition}
    Let $(X, \mathcal{U})$ is a Hausdorff space(separable space) if for any $x, y \in X$, there exists open sets $U, V \in \mathcal{U}$ such that $x \in U$, $y \in V$ and $U \cap V = \emptyset$. 
\end{definition}
\par \noindent \textbf{Remark} Finite complement topology in \ref{Not metric2} is not Hausdorff. If we take any two non-empty open sets $U, V$, then both $X \setminus U$ and $X \setminus V$ are finite. Thus, $X \setminus (U \cap V) = (X \setminus U) \cup (X \setminus V)$ is also finite, which means $U \cap V \neq \emptyset$. So it is not Hausdorff. 
\par \noindent \textbf{Remark} Also we see that \ref{Not metric1} is not Hausdorff. 

\begin{example}
    Let $X = \mathbb{N}$. Let $U_n = \{i\in X| i\le n\}$ for some $n$. Let $\mathcal{U} = \{\emptyset, X\} \cup \{U_n | n \in \mathbb{N}\}$. Then $(X, \mathcal{U})$ is a topological space. 
\par For finite Union, let $U_{n_1}, U_{n_2}, \ldots, U_{n_k} \in \mathcal{U}$, then 
    \begin{equation}
        \bigcup_{i=1}^k U_{n_i} = U_{\max\{n_1, n_2, \ldots, n_k\}} \in \mathcal{U}.
    \end{equation}
\par For infinite Union, let $\{U_{n_{\alpha}}\}_{\alpha \in A} \subseteq \mathcal{U}$, then 
    \begin{equation}
        \bigcup_{\alpha \in A} U_{n_{\alpha}} = \mathbb{N}
    \end{equation}
\par For finite Intersection, let $U_{n_1}, U_{n_2}, \ldots, U_{n_k} \in \mathcal{U}$, then 
    \begin{equation}
        \bigcap_{i=1}^k U_{n_i} = U_{\min\{n_1, n_2, \ldots, n_k\}} \in \mathcal{U}.
    \end{equation}
\par This set is not Hausdorff. 
\end{example}

\par Alternatively, topology can be defined by closed sets: $Y \subseteq X$ is closed if $X\setminus Y$ is open. So we can rewrite the definition of Topology space: 
\begin{enumerate}
    \item $\emptyset, X$ are closed.
    \item If $F_{\alpha}$ is closed for all $\alpha \in A$, then $\bigcap_{\alpha \in A} F_{\alpha}$ is closed.
    \item If $F_1, F_2, \ldots, F_n$ are closed, then $\bigcup_{i=1}^n F_i$ is closed.
\end{enumerate}
\par \noindent \textbf{Question} Is there an infinite collection of open sets such that their intersection is not open? 
\section{Zariski Topology} 
\par Let $X = \mathbb{C}^n$(or $\mathbb{K}^n$). $Y \subseteq \mathbb{C}^n$ is closed if $Y$ is a solution of 
\begin{equation}
    \begin{cases}
    f_1(x_1, x_2, \ldots, x_n) = 0 \\
    f_2(x_1, x_2, \ldots, x_n) = 0 \\
    \vdots \\
    f_m(x_1, x_2, \ldots, x_n) = 0\\
    \vdots 
    \end{cases}
\end{equation} 
\par $X$ is defined by $0 = 0$. $\emptyset$ is defined by $1 = 0$. Any points in $\mathbb{C}^n$ is defined by
\begin{equation}
    \begin{cases}
    x_1 - a_1 = 0 \\
    x_2 - a_2 = 0 \\
    \vdots \\
    x_n - a_n = 0\\
    \end{cases}
\end{equation}
A circle in $\mathbb{C}^2$ is defined by $x^2 + y^2 - 1 = 0$. 
\par Any intersection of closed sets is the set of solutions of all the polynomials that define each closed set, so it is still closed. 
\par Any union of finite closed sets is also closed since we can use the product of all polynomials that define each closed set to define the union. 
\par For $n = 1$, since any polynomial with degree at least 1 has finitely many roots, recalling the finite complement topology in \ref{Not metric2}, we see that Zariski topology on $\mathbb{C}$ is the finite complement topology. But this is not Hausdorff. 

\section{Basis of a topology} 
\begin{definition}
    Let $X$ be a set. A \textbf{basis} $\mathcal{B}$ is a is a set in $2^{X}$ such that
    \begin{enumerate}
        \item For any $x \in X$, $\exists B \in \mathcal{B}$ such that $x \in B$. 
        \item For any $x\in X$, if $x \in B_1$ and $x \in B_2$ where $B_1, B_2 \in \mathcal{B}$, then $\exists B_3 \in \mathcal{B}$ such that $x \in B_3 \subseteq B_1 \cap B_2$.
    \end{enumerate}
\end{definition}
\begin{property}
    The collection of all unions of elements in $\mathcal{B}$ forms a topology on $X$, called the topology \textbf{generated} by $\mathcal{B}$.
\end{property}
\par \noindent \textbf{Proof} 
\begin{enumerate}
    \item $\emptyset, X$ are in the topology generated by $\mathcal{B}$.
    \item Let $\{U_{\alpha}\}_{\alpha \in A}$ be a collection of open sets in the topology generated by $\mathcal{B}$. Then for each $\alpha$, we have $U_{\alpha} = \bigcup_{i \in I_{\alpha}} B_i$ where $B_i \in \mathcal{B}$. Thus,
    \begin{equation}
        \bigcup_{\alpha \in A} U_{\alpha} = \bigcup_{\alpha \in A} \bigcup_{i \in I_{\alpha}} B_i,
    \end{equation}
    which is still a union of elements in $\mathcal{B}$. So $\bigcup_{\alpha \in A} U_{\alpha}$ is in the topology generated by $\mathcal{B}$.
    \item Let $U_1, U_2\in \mathcal{U}$ be two open sets in the topology generated by $\mathcal{B}$. Then let $x\in U_1 \cap U_2$, we have $\exists B_1, B_2 \in \mathcal{B}$ such that $x \in B_1 \subseteq U_1$ and $x \in B_2 \subseteq U_2$. By the second property of basis, there exists $B_3 \in \mathcal{B}$ such that $x \in B_3 \subseteq B_1 \cap B_2 \subseteq U_1 \cap U_2$. Similarly, for any $x \in U_1 \cap U_2$, we can find such $B_3$. Thus, $U_1 \cap U_2$ is also in the topology generated by $\mathcal{B}$. 
\end{enumerate}
\begin{example}\label{basis1}
    For a metric space $(X, d)$, the collection of all open balls $\{B(x, r) | x \in X, r > 0\}$ is a basis of the topology induced by $d$.
\end{example}
\begin{example}\label{basis2}
    All the rectangles in $\mathbb{R}^2$ with sides parallel to the axes form a basis of the standard topology on $\mathbb{R}^2$.
\end{example}

\begin{example}
    Let $X$ be any set. Let $\mathcal{B} = \{\{x\} | x \in X\}$. Then $\mathcal{B}$ is a basis of the discrete topology on $X$.
\end{example}

\begin{theorem}
    If $\mathcal{B}$ is a basis of a topology $\mathcal{U}$ on $X$, then for any open set $U \in \mathcal{U}$, we have $U = \bigcup_{B \in \mathcal{B}, B \subseteq U} B$. (The Proof is based on the wrong definition of basis in the previous version, but the right version is not hard. )
\end{theorem}
\par \noindent \textbf{Proof}
\par ``$\subseteq$'': For any $x \in U$, since $\mathcal{B}$ is a basis, there exists $B \in \mathcal{B}$ such that $x \in B$. There exists $B' \in \mathcal{B}$ such that $x \in B' \subseteq B \cap U$. Thus, $B' \subseteq U$ and $x \in \bigcup_{B \in \mathcal{B}, B \subseteq U} B$.
\par ``$\supseteq$'': For any $x \in \bigcup_{B \in \mathcal{B}, B \subseteq U} B$, there exists $B \in \mathcal{B}$ such that $x \in B \subseteq U$. Thus, $x \in U$.
\begin{example}
    \begin{enumerate}
        \item Balls in $d_1, d_2, d_{\infty}$ metrics in $\mathbb{R}^n$ form bases of the standard topology on $\mathbb{R}^n$. 
        \item $\{U_{\epsilon}(x)| \epsilon > 0, x\in \mathbb{Q}^n\}$is a countable basis of the standard topology on $\mathbb{R}^n$. 
    \end{enumerate}
\end{example}
\par \noindent \textbf{Exercise} Find a metric space not admitting countable basis of topology. 

\begin{lemma}
    Let $X$ be a topological space. Let $\mathcal{C}$ be a collection of open sets of $X$ such that: let $x$ be a point in $X$, then for any open set $U$ containing $x$, there exists $C \in \mathcal{C}$ such that $x \in C \subseteq U$. Then $\mathcal{C}$ is a basis of the topology on $X$. 
\end{lemma}
\par \noindent \textbf{Proof} 
\begin{enumerate}
    \item For any $x \in X$, since $X$ is open and contains $x$, there exists $C \in \mathcal{C}$ such that $x \in C \subseteq X$. 
    \item Let $x \in C_1 \cap C_2$ where $C_1, C_2 \in \mathcal{C}$. Since $C_1 \cap C_2$ is open and contains $x$, there exists $C_3 \in \mathcal{C}$ such that $x \in C_3 \subseteq C_1 \cap C_2$.
    \item Let $\mathcal{T}$ be the topology generated by $\mathcal{C}$. Let $\mathcal{U}$ be the initial topology on $X$. For any $U \in \mathcal{U}$, for any $x \in U$, there exists $C \in \mathcal{C}$ such that $x \in C \subseteq U$. So, \begin{equation}
        U = \bigcup_{C \in \mathcal{C}, C \subseteq U} C,
    \end{equation}
    which means $U \in \mathcal{T}$. Thus, $\mathcal{U} \subseteq \mathcal{T}$. 
    \item And we also have $\mathcal{T} \subseteq \mathcal{U}$ because: since $\mathcal{C} \subseteq \mathcal{U}$, any union of elements in $\mathcal{C}$ is also in $\mathcal{U}$. Thus, $\mathcal{T} \subseteq \mathcal{U}$.
    \item Combining the above two results, we have $\mathcal{T} = \mathcal{U}$.
\end{enumerate}

\begin{definition}
    If $\mathcal{T}, \mathcal{U}$ are two topologies on $X$, we say that $\mathcal{T}$ is \textbf{finer} than $\mathcal{U}$ (or $\mathcal{U}$ is \textbf{coarser} than $\mathcal{T}$) if $\mathcal{U} \subseteq \mathcal{T}$.
\end{definition}

\begin{example}
    Let $X$ be any set. The discrete topology on $X$ is the finest topology on $X$. The anti-discrete topology on $X$ is the coarsest topology on $X$. 
\end{example}

\begin{lemma}
    Let $\mathcal{B}, \mathcal{C}$ be two bases of topologies $\mathcal{T}, \mathcal{U}$ on $X$ respectively. Then the following are equivalent:
    \begin{enumerate}
        \item $\mathcal{U}$ is finer than $\mathcal{T}$. 
        \item For any $x \in X$, for any $B \in \mathcal{B}$ containing $x$, there exists $C \in \mathcal{C}$ such that $x \in C \subseteq B$.
    \end{enumerate}
\end{lemma}

\par \noindent \textbf{Proof} 
\begin{enumerate}
    \item ``(1) $\Rightarrow$ (2)'': Let $x \in X$. Let $B \in \mathcal{B}$ such that $x \in B$. Since $\mathcal{U}$ is finer than $\mathcal{T}$, we have $B \in \mathcal{U}$. Since $\mathcal{C}$ is a basis of $\mathcal{U}$, there exists $C \in \mathcal{C}$ such that $x \in C \subseteq B$.
    \item ``(2) $\Rightarrow$ (1)'': Let $U \in \mathcal{T}$. For any $x \in U$, since $\mathcal{B}$ is a basis of $\mathcal{T}$, there exists $B \in \mathcal{B}$ such that $x \in B \subseteq U$. By (2), there exists $C \in \mathcal{C}$ such that $x \in C \subseteq B \subseteq U$. Thus, for any $x \in U$, we can find such $C$. By the previous lemma($\mathcal{C}$ is a basis of $\mathcal{U}$), we have $U \in \mathcal{U}$. Therefore, $\mathcal{U}$ is finer than $\mathcal{T}$.
\end{enumerate}

\begin{example}
    By the above lemma, we see that the topology generated by the basis in \ref{basis1} is finer than the topology generated by the basis in \ref{basis2}. And similarly, the topology generated by the basis in \ref{basis2} is finer than the topology generated by the basis in \ref{basis1}. Thus, they are the same topology.
\end{example}
\begin{example}
    [Lower limit topology on $\mathbb{R}$]
    Let $\mathcal{B} = \{[a, b) | a < b, a, b \in \mathbb{R}\}$. Then $\mathcal{B}$ is a basis of a topology on $\mathbb{R}$, called the \textbf{lower limit topology} called $\mathbb{R}_l$.
    \par One can show that the lower limit topology is finer than the standard topology on $\mathbb{R}$ since for $a < x < b$, we have $x \in [x, b) \subseteq (a, b)$. But the lower limit topology is not equal to the standard topology since take $0\in [0, 1)$, there is no open interval $(a, b)$ such that $0 \in (a, b) \subseteq [0, 1)$.
    \par We say that $\mathbb{R}_l$ is strictly finer than the standard topology on $\mathbb{R}$. 
\end{example}
\par \noindent \textbf{Exercise} Prove that $\mathbb{R}_l$ does not have a countable basis. 
\begin{example}
    [Order Topology] 
    Let $X$ be a simply ordered set. 
    \begin{equation}
        \mathcal{B} = \{(a, b) | a < b, a, b \in X\} \cup \{[a_0, b)| a_0 < b\} \cup \{(a, b_0] | a < b_0\}
    \end{equation} 
    where $a_0$ is the smallest element of $X$ if it exists, and $b_0$ is the largest element of $X$ if it exists. Then $\mathcal{B}$ is a basis of a topology on $X$, called the \textbf{order topology}.
\end{example}
\par \noindent \textbf{Remark} Let $(X, \mathcal{U})$, $(Y, \mathcal{Y})$ be two topological spaces. Then $\mathcal{U}\times \mathcal{V}$ may not be a topology on $X \times Y$. For example, the union of two rectangles may not be a rectangle. The correct idea is to define the basis of the product topology as:
\begin{equation}
    \mathcal{B} = \{U \times V | U \in \mathcal{U}, V \in \mathcal{V}\}.
\end{equation}
Furthermore, we see that for $U_1, U_2\in \mathcal{U}$ and $V_1, V_2 \in \mathcal{V}$, we have
\begin{equation}
    (U_1 \times V_1) \cap (U_2 \times V_2) = (U_1 \cap U_2) \times (V_1 \cap V_2),
\end{equation}
\begin{definition}
    The topology on $X\times Y$ defined by basis $\mathcal{U}\times \mathcal{V}$ is called the product topology on $X\times Y$. 
\end{definition}
\begin{theorem}
    Let $\mathcal{B} \subseteq \mathcal{U}$ be a basis of $\mathcal{U}$ and let $\mathcal{C} \subseteq \mathcal{V}$ be a basis of $\mathcal{V}$. Then $\mathcal{D} = \{B \times C | B \in \mathcal{B}, C \in \mathcal{C}\}$ is a basis of the product topology on $X \times Y$.
\end{theorem}

\par \noindent \textbf{Proof} 
\par We need to show: for every open set $W \subseteq X\times Y$ and for every $(x, y) \in W$, there exists $B\times C$ with $B \in \mathcal{B}$, $C \in \mathcal{C}$ such that $(x, y) \in B \times C \subseteq W$.
\par Since $W$ is open in the product topology, there exists $U \in \mathcal{U}$ and $V \in \mathcal{V}$ such that $(x, y) \in U \times V \subseteq W$. Since $\mathcal{B}$ is a basis of $\mathcal{U}$, there exists $B \in \mathcal{B}$ such that $x \in B \subseteq U$. Since $\mathcal{C}$ is a basis of $\mathcal{V}$, there exists $C \in \mathcal{C}$ such that $y \in C \subseteq V$. Thus, we have $(x, y) \in B \times C \subseteq U \times V \subseteq W$.

\begin{example}
    Let $\pi_1: X \times Y \to X$ and $\pi_2: X \times Y \to Y$ be the projection maps defined by $\pi_1(x, y) = x$ and $\pi_2(x, y) = y$. Then we have 
    \begin{equation}
        \pi_1^{-1}(U) = U \times Y \text{ for any } U \in \mathcal{U},
    \end{equation}
    \begin{equation}
        \pi_2^{-1}(V) = X \times V \text{ for any } V \in \mathcal{V}.
    \end{equation}
    which means that both $\pi_1$ and $\pi_2$ are continuous. And we have 
    \begin{equation}
        U\times V = \pi_1^{-1}(U) \cap \pi_2^{-1}(V).
    \end{equation}
    is open in the product topology. And we have 
    \begin{equation}
        S = \{\pi_1^{-1}(U)| U \in \mathcal{U}\} \cup \{\pi_2^{-1}(V) | V \in \mathcal{V}\} 
    \end{equation}
    is a subbasis of the product topology on $X \times Y$(Note: We have not defined subbasis). 
\end{example}

\end{document}