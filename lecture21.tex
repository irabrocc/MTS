% 声明为子文件,指定主文件
\documentclass[main.tex]{subfiles}

\begin{document}
\pagestyle{plain}
\setcounter{chapter}{20}

\chapter{Later}
\label{chap:chapter21}

%\par Let us consider the simplest case: the product of two %compact spaces $X_1 \times X_2$. Suppose that $\mathcal{A}$ %is a collection of closed subsets of $X_1 \times X_2$ that %has the finite intersection property. Consider the %projection map $\pi_1: X_1 \times X_2 \rightarrow X_1$. The %collection 
%\begin{equation}
%    \{\pi_1(A) | A\in \mathcal{A}\}
%\end{equation}
%of subsets of $X_1$ also has the finite intersection %property, and so does the collection of their closures %$\overline{\pi_1(A)}$(Closedness might not be preserved %under projection). Compactness of $X_1$ guarantees that the %intersection of all the sets $\overline{\pi_1(A)}$ is not %empty. Let us choose the point $x_1$ belonging to this %intersection. Similarly, let us choose a point $x_2$ %belonging to the sets $\overline{\pi_2(A)}$. 
%\par We want to draw the conclusion that the point %$x_1\times x_2$ lies in $\bigcap_{A\in \mathcal{A}}$, for %then out theorem would be proved.  

\par \noindent \textbf{Idea} For the idea of this lecture, refer to Munkres' Topology, P230 or pdf P247. 

\begin{definition}
    Let $(A, \leq)$ be a partially ordered set. A subset $B \subseteq A$ is called a chain if for any $a, b \in B$, we have either $a \leq b$ or $b \leq a$.
\end{definition}


\begin{theorem}
    [Zorn's lemma]
    Let $(A, \leq)$ be a partially ordered set. If every chain in $A$ has an upper bound in $A$, then $A$ contains at least one maximal element.
\end{theorem}

\begin{lemma}\label{lemma:maximal}
    Let $X$ be a set; let $\mathcal{A}$ be a collection of subsets of $X$ having the finite intersection property. Then there is a collection $\mathcal{D}$ of subsets of $X$ such that $\mathcal{D}$ contains $\mathcal{A}$, and $\mathcal{D}$ has the finite intersection property, and no collections of subsets of $X$ properly containing $\mathcal{D}$ has the finite intersection property. 
\end{lemma}
\par \noindent \textbf{Proof} 
\par For purposes of this proof, we shall call a set whose elements are collections of subsets of $X$ a "superset" and shall denote it by an outline letter. To summarize the notation: 
\begin{itemize}
    \item $c$ is an element of $X$;
    \item $C$ is a subset of $X$; 
    \item $\mathcal{C}$ is a collection of subsets of $X$; 
    \item $\mathbb{C}$ is a superset whose elements are collections of subsets of $X$. 
\end{itemize} 

\par Now by hypothesis, $\mathcal{A}$ is a collection of subsets of $X$ having the finite intersection property. Let $\mathbb{A}$ be the superset consisting of all collections of subsets of $X$ such that $\forall \mathcal{B} \in \mathbb{A}$, we have $\mathcal{B} \supseteq \mathcal{A}$ and $\mathcal{B}$ has the finite intersection property. To prove the lemma, we need to show that $\mathbb{A}$ contains a maximal element with respect to the partial order $\subseteq$. 
\par In order to apply Zorn's lemma, we must show that if $\mathbb{B}$ is a "subsuperset" of $\mathbb{A}$ that is a chain, then $\mathbb{B}$ has an upper bound in $\mathbb{A}$. Let 
\begin{equation}
    \mathcal{C} = \bigcup_{\mathcal{B} \in \mathbb{B}} \mathcal{B}.
\end{equation}
Certainly $\mathcal{C}$ contains $\mathcal{A}$, since each $\mathcal{B} \in \mathbb{B}$ contains $\mathcal{A}$. Let $C_1, C_2, \ldots, C_n$ be finitely many elements of $\mathcal{C}$. Then for each $i = 1, 2, \ldots, n$, there exists $\mathcal{B}_i \in \mathbb{B}$ such that $C_i \in \mathcal{B}_i$. Since $\mathbb{B}$ is a chain, there exists $\mathcal{B}_j \in \mathbb{B}$ such that $\mathcal{B}_i \subseteq \mathcal{B}_j$ for all $i = 1, 2, \ldots, n$. Thus $C_i \in \mathcal{B}_j$ for all $i = 1, 2, \ldots, n$. Since $\mathcal{B}_j$ has the finite intersection property, we have 
\begin{equation}
    C_1 \cap C_2 \cap \ldots \cap C_n \neq \varnothing.
\end{equation}
Thus $\mathcal{C}$ has the finite intersection property. And so $\mathcal{C} \in \mathbb{A}$. Moreover, for each $\mathcal{B} \in \mathbb{B}$, we have $\mathcal{B} \subseteq \mathcal{C}$. Thus $\mathcal{C}$ is an upper bound of $\mathbb{B}$ in $\mathbb{A}$. By Zorn's lemma, $\mathbb{A}$ contains a maximal element, which we denote by $\mathcal{D}$. This completes the proof of the lemma. 
\mbox{} \\ \null \hfill $\blacksquare$ 
\begin{lemma}
    Let $X$ be a set; let $\mathcal{D}$ be a collection of subsets of $X$ that is maximal with respect to the finite intersection property. Then 
    \begin{enumerate}
        \item Any finite intersection of elements of $\mathcal{D}$ is an element of $\mathcal{D}$; 
        \item If $A$ is a subset of $X$ that intersects every element of $\mathcal{D}$, then $A$ is an element of $\mathcal{D}$. 
    \end{enumerate}
\end{lemma}

\par \noindent \textbf{Proof to 1} 
\par Let $B$ be any finite intersection of elements of $\mathcal{D}$. Let $\mathcal{E} = \mathcal{D} \cup \{B\}$. We can show that $\mathcal{E}$ has the finite intersection property(Check it). By maximality of $\mathcal{D}$, we have $\mathcal{E} = \mathcal{D}$, so $B \in \mathcal{D}$. 
\par \noindent \textbf{Proof to 2} 
\par Let $A$ be a subset of $X$ that intersects every element of $\mathcal{D}$. Let $\mathcal{E} = \mathcal{D} \cup \{A\}$. Take finitely many elements from $\mathcal{E}$. If none of them is $A$, then their intersection is not empty because $\mathcal{D}$ has the finite intersection property, otherwise, it is of the form 
\begin{equation}
    A \cap D_1 \cap D_2 \cap \ldots \cap D_n,
\end{equation}
Now, $D_1 \cap D_2 \cap \ldots \cap D_n \in \mathcal{D}$ by part 1, so $A$ intersects it by hypothesis. Thus the intersection is not empty. So $\mathcal{E}$ has the finite intersection property. By maximality of $\mathcal{D}$, we have $\mathcal{E} = \mathcal{D}$, so $A \in \mathcal{D}$.
\mbox{} \\ \null \hfill $\blacksquare$ 
\begin{theorem}
    [Tychonoff theorem] 
    Any arbitrary product of compact spaces is compact in the product topology. 
\end{theorem}
\par \noindent \textbf{Proof} 
\par Let  
\begin{equation}
    X = \prod_{\alpha \in J} X_{\alpha},
\end{equation}
where each $X_{\alpha}$ is compact. Let $\mathcal{A}$ be a collection of subsets of $X$ having the finite intersection property. We prove that the intersection $\bigcap_{A \in \mathcal{A}} \bar{A}$
is non-empty. Compactness of $X$ follows. 
\par By \ref{lemma:maximal}, we choose $\mathcal{D}$ containing $\mathcal{A}$ that is maximal with respect to the finite intersection property. It suffices to show that the intersection 
\begin{equation}
    \bigcap_{D \in \mathcal{D}} \bar{D} 
\end{equation}
is non-empty. 
\par Given any $\alpha \in J$, we consider the collection 
\begin{equation}
    \{\pi_{\alpha}(D) | D\in \mathcal{D}\}
\end{equation}
of subsets of $X_{\alpha}$. This collection has the finite intersection property because $\mathcal{D}$ does. By compactness of $X_{\alpha}$, we can choose a point $x_{\alpha} \in X_{\alpha}$ such that 
\begin{equation}
    x_{\alpha} \in \bigcap_{D \in \mathcal{D}} \overline{\pi_{\alpha}(D)}.
\end{equation}
\par Now the point $x = (x_{\alpha})$ is defined in $X$. 
\par Let $U_{\beta}$ be a neighborhood of $x_{\beta}$ in $X_{\beta}$. Since $x_{\beta} \in \overline{\pi_{\beta}(D)}$, we have that $\exists y\in D, \pi_{\beta}(y) \in U_{\beta} \cap \pi_{\beta}(D)$. Thus $y \in \pi_{\beta}^{-1}(U_{\beta}) \cap D$. Since this is true for any $D \in \mathcal{D}$, by the previous lemma, we have $\pi_{\beta}^{-1}(U_{\beta}) \in \mathcal{D}$. So every subbasis element containing $x$ is in $\mathcal{D}$. And then it follows from the same lemma that every basis(finite intersection of subbasis) element containing $x$ is in $\mathcal{D}$. Thus any open neighborhood of $x$ intersects every $D \in \mathcal{D}$. Therefore, $x \in \bar{D}$ for all $D \in \mathcal{D}$. 
\mbox{} \\ \null \hfill $\blacksquare$  


\end{document}