% 声明为子文件,指定主文件
\documentclass[main.tex]{subfiles}

\begin{document}
\pagestyle{plain}
\setcounter{chapter}{15}

\chapter{Later}
\label{chap:chapter16}

\begin{definition}
    [def1]
    $X$ is compact if any open covering admits a finite subcovering. 
\end{definition}

\begin{definition}
    [def2: Frechet compactness]
    $X$ is limit point compact if every infinite subset of $X$ has a limit point in $X$. 
\end{definition}

\begin{definition}
    [def3: sequential compactness/Bozano-Weierstrass compactness]
    $X$ is sequentially compact if every sequence $\{x_n\} \subseteq X$ has a convergent subsequence converging to a point in $X$.
\end{definition}

\begin{theorem}
    If $X$ is a metric space, then $X$ is compact if and only if it is limit point compact if and only if it is sequentially compact.
\end{theorem}
\par \noindent \textbf{Proof} 
\par The proof is quite long and technical. We will only some directions in the following. 

\begin{theorem}
    For an arbitrary topological space, compactness implies limit point compactness. The converse is not true in general.
\end{theorem}
\par \noindent \textbf{Proof} 
\par Let $X$ be compact, and let $A \subseteq X$ be infinite. 
\par We assume that $A$ has no limit point. Then $A = \bar{A}$. 
So $X\setminus A$ is open. Furthermore, for each $a \in A$, we can choose a neighborhood $U_a$ of $a$ such that $U_a \cap (A\setminus \{a\}) = \emptyset$.
\par Then $\{U_a: a \in A\} \cup \{X\setminus A\}$ is an open covering of $X$. Since $X$ is compact, there exists a finite subcovering, say $\{U_{a_1}, U_{a_2}, \cdots, U_{a_n}\} \cup \{X\setminus A\}$. Thus,
\[ X = \bigcup_{i=1}^n U_{a_i} \cup (X\setminus A) = (X\setminus A)\cup \{a_1, a_2, \cdots, a_n\}, \]
which implies that $A$ is finite, a contradiction. Hence, $A$ has a limit point in $X$. 
\mbox{} \\ \null \hfill $\blacksquare$ 

\begin{example}
    [Counterexample for the converse]
    Let $Y = \{p, q\}$ with anti-dicrete topology $J_Y = \{\emptyset, Y\}$. Let $X = \mathbb{N} \times Y$ with the product topology where $\mathbb{N}$ has the discrete topology. Then every non-empty set $A \subseteq X$ has a limit point. Because if $(n, p) \in A$, then any open set containing $(n, q)$ intersects $A$ at $(n, p)$. So $X$ is limit point compact. However, $X$ is not compact. Because $\{ \{n\} \times Y: n \in \mathbb{N} \}$ is an open covering of $X$ which admits no finite subcovering. 
\end{example}

\begin{example}
    [Limit point compact but not sequentially compact]
    Let $X$ be defined as the above example. Consider the sequence $\{(n, p)\}_{n=1}^{\infty}$. This sequence has no convergent subsequence. Because for any point $(m, p)$, the open set $\{m\} \times Y$ contains only finitely many terms of the sequence; for any point $(m, q)$, the open set $\{m\} \times Y$ also contains only finitely many terms of the sequence. So $X$ is not sequentially compact.
\end{example}

\begin{theorem}
    For a first-countable topological space $X$, limit point compactness implies sequential compactness.
\end{theorem}

\par \noindent \textbf{Proof} 
\par Take a sequence $\{x_n\} \subseteq X$. 
\par If the set of values $\{x_n: n \in \mathbb{N}\}$ is finite, then there exists a value $x$ that appears infinitely many times in the sequence. So the subsequence constantly equal to $x$ converges to $x$.
\par Suppose the set of values $\{x_n: n \in \mathbb{N}\}$ is infinite. Since $X$ is first-countable, we can construct a countable basis $\{U_k\}$ at a limit point $a$ such that 
\begin{equation}
    x_{n_1} \in U_1, x_{n_2} \in U_2, \cdots, x_{n_k} \in U_k, \cdots
\end{equation}
and 
\begin{equation}
    U_1 \supseteq U_2 \supseteq \cdots \supseteq U_k \supseteq \cdots
\end{equation}
That is for each open set $U \ni a$ there exists $N$ such that $\forall k \ge N$, $U_k \subseteq U$. Then $x_{n_k} \in U_k \subseteq U$ for all $k \ge N$. So $x_{n_k} \to a$. \hfill $\blacksquare$


\par \noindent \textbf{Remark} For topological spaces, we have the following facts: 

\begin{enumerate}
    \item Compactness $\Rightarrow$ Limit Point Compactness
    \item Sequential Compactness $\Rightarrow$ Limit Point Compactness 
    \item But other implications are not true in general.
\end{enumerate}

\begin{theorem}
    Sequential compactness implies compactness for metric spaces.
\end{theorem}
\par \noindent \textbf{Proof} This is harder than other directions. 


\par \noindent \textbf{Exercise} Prove that $X$ is sequentially compact, then it is limit point compact. 

\section{Locally Compact} 

\begin{definition}
    $X$ is said to be locally compact at $x\in X$ if there exists an neighborhood $U$ of $x$ and there exists a compact subspace $C$ of $X$ containing $U$. $X$ is locally compact if it's locally compact at each $x\in X$. 
\end{definition}

\begin{example}
    $\mathbb{R}$ is locally compact. Because for any $x\in \mathbb{R}$, take $U = (x - 1, x + 1)$ and $C = [x - 1, x + 1]$ which is compact. 
\end{example}

\begin{example}
    $\mathbb{R}^n$ is locally compact. 
\end{example}

\begin{example}
    [non example] 
    $\mathbb{Q} \subseteq \mathbb{R}$ is not locally compact.
\end{example}

\par \noindent \textbf{Exercise} Show the above example. 

\begin{example}
    [non example] 
    $\mathbb{R}^{\omega}$ with product topology is not locally compact. Let $U = (a_1, b_1) \times (a_2, b_2) \times \cdots$. If $U \subseteq C$ where $C$ is compact, then $\overline{U} = [a_1, b_1] \times [a_2, b_2] \times \cdots \subseteq C$. We know a closed subset of a compact set is compact, so $\overline{U}$ is compact. But this is not true. 
\end{example}

\par \noindent \textbf{Exercise} Show the above example. 

\begin{theorem}
    Let $X$ be a topological space. Then $X$ is Hausdorff locally compact if and only if there exists a spcace $Y$ such that
    \begin{enumerate}
        \item $X$ is a subspace of $Y$.
        \item $Y \setminus X$ contains exactly one point $\{p\}$.
        \item $Y$ is compact Hausdorff. 
    \end{enumerate}
    This $Y$ is unique in the following sense: If $Y, Y'$ are two spaces with the above properties and $Y = X \cup \{p\}, Y' = X \cup \{q\}$, then there exists a homeomorphism $h: Y \to Y'$ such that $h(x) = x$ for all $x \in X$ and $h(p) = q$.
\end{theorem}
\par \noindent \textbf{Proof} 
\par \noindent \textbf{Uniqueness} Let $Y = X \cup \{p\}$, $Y' = X \cup \{q\}$ satisfies the above properties. Define $h: Y \to Y'$ as follows: $h(x) = x$ for all $x \in X$ and $h(p) = q$. We show that $h$ is continuous. But the function is symmetric, so it is enough to show that $h(U)$ is open in $Y'$ for all open set $U$ in $Y$. 
\par Take $U$ be open. There are two cases. 
\par If $U \subseteq X$, then we are done. 
\par Suppose $p\in U$. Then $C = Y \setminus U$ is closed in $Y$. So $C$ is compact. 
\par \noindent \textbf{Construction} We introduce the topology on $Y = X \cup \{\infty\}$ as follows: 
\par There are two types of open sets in $Y$: 
\begin{enumerate}
    \item $U \subseteq X \subseteq Y$ is open in $X$.
    \item $U = Y \setminus C$ where $C$ is compact. 
\end{enumerate}

We need to check that this is a topology. 
\par For the intersection of two open sets of type 1, we have $U_1 \cap U_2$ is open in $X$ thus is open in $Y$. 
\par For the intersection of type 1 and type 2, we have $U_1 \cap (Y \setminus C) = U\cap(X \setminus C)$ which is the union of open sets in $X$ thus is open in $Y$. 
\par For the intersection of two open sets of type

\par It remains to showw $Y$ is compact Hausdorff and if $X \subseteq Y$ with $Y$ satisfied the three properties, then $X$ is locally compact. 
\par We can show that $X$ is a subspace of $Y$ because: 
\begin{enumerate}
    \item If $ U \subseteq X$ and $\infty \notin U$, then 
    \item $(Y\setminus C) \cap X = X \setminus C$ where $C$ is compact in $X$.
\end{enumerate}

\par We show that $Y$ is compact. 
\par Let $\mathcal{A}$ be an open covering of $Y$. Then there exists compact $C$ such that $Y \setminus C \in \mathcal{A}$. The rest of the covering $\mathcal{A}' = \mathcal{A} \setminus \{Y \setminus C\}$ is an open covering of $C$. Since $C$ is compact, there exists a finite subcovering of $\mathcal{A}'$ covering $C$. Thus, adding $Y \setminus C$ gives a finite subcovering of $\mathcal{A}$ covering $Y$. So $Y$ is compact.
\par We show that $Y$ is Hausdorff.
\par Take two distinct points $x, y \in Y$. There are two cases.
\begin{enumerate}
    \item If $x, y \in X$, since $X$ is Hausdorff, there exist disjoint open sets $U, V$ in $X$ such that $x \in U, y \in V$. Then $U, V$ are open in $Y$ and disjoint.
    \item If one of them is $\infty$, say $y = \infty$. Since $X$ is locally compact at $x$, there exists an open neighborhood $U$ of $x$ and a compact set $C$ such that $U \subseteq C$. Then $V = Y \setminus C$ is an open neighborhood of $\infty$. Clearly, $U \cap V = \emptyset$.
\end{enumerate}
\par We prove the other direction.
\par To be done. 

\mbox{} \\ \null \hfill $\blacksquare$ 


\begin{theorem}
    Let $X$ be a Hausdorff space. Then $X$ is locally compact if and only if for any $x\in X$ and any neighborhood $U$ of $x$, there exists a neighborhood $V$ of $x$ such that $\bar{V}$ is compact and $\bar{V} \subseteq U$.
\end{theorem}

\par \noindent \textbf{Proof} 
\par $\Leftarrow$ $x\in V \subseteq \bar{V} \subseteq U$ and $\bar{V}$ is compact. So $X$ is locally compact at $x$.
\par $\Rightarrow$ Let $Y$ be the one-point compactification(compact and Hausdorff) of $X$. Let $U$ be a neighborhood of $x$ in $Y$. Then $Y\setminus U:= C$ is a closed subset of $Y$. So $C$ is compact. By lemma 12.0.2, there exists two open sets $V, W$ in $Y$ such that $x \in V$, $C \subseteq W$ and $V \cap W = \emptyset$. Then $\bar{V}$ is compact and $\bar{V} \cap C = \emptyset$. So $\bar{V} \subseteq U$. \hfill $\blacksquare$

\begin{corollary}
    Let $X$ be locally compact Hausdorff space. $A \subseteq X$ is open or closed. Then $A$ is locally compact.
\end{corollary}

\par \noindent \textbf{Proof} 
\par Let $A \subseteq X$ be closed. Given $x\in A$, since $X$ is locally compact, there exists an open neighborhood $U$ of $x$ in $X$ and a compact set $C$ such that $x \in U \subseteq C$. Then $U \cap A$ is an open neighborhood of $x$ in $A$ and $C \cap A$ is compact in $A$. So $A$ is locally compact at $x$.

\par Let $A \subseteq X$ be open. Given $x\in A$, since $X$ is locally compact and $X$ is Hausdorff, by the previous theorem, there exists a neighborhood $V$ of $x$ such that $\bar{V}$ is compact and $\bar{V} \subseteq A$. So $A$ is locally compact at $x$. \hfill $\blacksquare$

\begin{corollary}
    $X$ is locally compact Hausdorff if and only if $X$ is homeomorphic to an open subspace of a compact Hausdorff space. 
\end{corollary}

\par \noindent \textbf{Exercise} Show the above corollary. 

\section{Urysohn's Metrization Theorem}
\begin{theorem}
    Every $X$ that is regular(T3) and second-countable is metrizable.
\end{theorem}

\section{Countability} 
\begin{definition}
    $X$ is first countable if for every $x\in X$, there exists a countable basis at $x$. That is, there exists $\mathcal{B} = \{B_n: n \in \mathbb{N}\}$ a collection of open sets containing $x$ such that for any open set $U$ containing $x$, there exists $B_n \in \mathcal{B}$ such that $B_n \subseteq U$.
\end{definition}

\begin{definition}
    $X$ is second countable if there exists a countable basis $\mathcal{B} = \{B_n: n \in \mathbb{N}\}$ for the topology of $X$. That is, for every $x$ and every open set $U$ containing $x$, there exists $B_n \in \mathcal{B}$ such that $x \in B_n \subseteq U$. 
\end{definition}

\par \noindent \textbf{Exercise} $\mathbb{R}^n$ with standard topology, $B_n = \{U_{\epsilon}(x)| x\in \mathbb{Q}^n, \epsilon \in \mathbb{Q}_{>0}\}$. $B_n$ is a countable basis. So $\mathbb{R}^n$ is second countable. Show the details. 

\par \noindent \textbf{Exercise} Show that if $X_n$'s are first(second) countable, then $\prod X_n$ with product topology is first(second) countable.

\begin{theorem}
    Let $X$ be second countable. Then 
    \begin{enumerate}
        \item Every open cover of $X$ has a countable subcovering($X$ is Lindelöf space).
        \item There is a countable subset $A \subseteq X$ such that $\bar{A} = X$($A \subseteq X$ is dense). 
    \end{enumerate}
\end{theorem}

\par \noindent \textbf{Proof} 

\end{document}