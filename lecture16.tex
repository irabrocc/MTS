% 声明为子文件,指定主文件
\documentclass[main.tex]{subfiles}

\begin{document}
\pagestyle{plain}
\setcounter{chapter}{15}

\chapter{Later}
\label{chap:chapter16}

\begin{definition}
    [def1]
    $X$ is compact if any open covering admits a finite subcovering. 
\end{definition}

\begin{definition}
    [def2: Frechet compactness]
    $X$ is limit point compact if every infinite subset of $X$ has a limit point in $X$. 
\end{definition}

\begin{definition}
    [def3: sequential compactness/Bozano-Weierstrass compactness]
    $X$ is sequentially compact if every sequence $\{x_n\} \subseteq X$ has a convergent subsequence converging to a point in $X$.
\end{definition}

\begin{theorem}
    If $X$ is a metric space, then $X$ is compact if and only if it is limit point compact if and only if it is sequentially compact.
\end{theorem}
\par \noindent \textbf{Proof} 
\par The proof is quite long and technical. We will only some directions in the following. 

\begin{theorem}
    For an arbitrary topological space, compactness implies limit point compactness. The converse is not true in general.
\end{theorem}
\par \noindent \textbf{Proof} 
\par Let $X$ be compact, and let $A \subseteq X$ be infinite. 
\par We assume that $A$ has no limit point. Then $A = \bar{A}$. 
So $X\setminus A$ is open. Furthermore, for each $a \in A$, we can choose a neighborhood $U_a$ of $a$ such that $U_a \cap (A\setminus \{a\}) = \emptyset$.
\par Then $\{U_a: a \in A\} \cup \{X\setminus A\}$ is an open covering of $X$. Since $X$ is compact, there exists a finite subcovering, say $\{U_{a_1}, U_{a_2}, \cdots, U_{a_n}\} \cup \{X\setminus A\}$. Thus,
\[ X = \bigcup_{i=1}^n U_{a_i} \cup (X\setminus A) = (X\setminus A)\cup \{a_1, a_2, \cdots, a_n\}, \]
which implies that $A$ is finite, a contradiction. Hence, $A$ has a limit point in $X$. 
\mbox{} \\ \null \hfill $\blacksquare$ 

\begin{example}
    [Counterexample for the converse]
    Let $Y = \{p, q\}$ with anti-dicrete topology $J_Y = \{\emptyset, Y\}$. Let $X = \mathbb{N} \times Y$ with the product topology where $\mathbb{N}$ has the discrete topology. Then every non-empty set $A \subseteq X$ has a limit point. Because if $(n, p) \in A$, then any open set containing $(n, q)$ intersects $A$ at $(n, p)$. So $X$ is limit point compact. However, $X$ is not compact. Because $\{ \{n\} \times Y: n \in \mathbb{N} \}$ is an open covering of $X$ which admits no finite subcovering. 
\end{example}

\begin{example}
    [Limit point compact but not sequentially compact]
    Let $X$ be defined as the above example. Consider the sequence $\{(n, p)\}_{n=1}^{\infty}$. This sequence has no convergent subsequence. Because for any point $(m, p)$, the open set $\{m\} \times Y$ contains only finitely many terms of the sequence; for any point $(m, q)$, the open set $\{m\} \times Y$ also contains only finitely many terms of the sequence. So $X$ is not sequentially compact.
\end{example}

\begin{theorem}
    For a first-countable topological space $X$, limit point compactness implies sequential compactness.
\end{theorem}

\par \noindent \textbf{Proof} 
\par Take a sequence $\{x_n\} \subseteq X$. 
\par If the set of values $\{x_n: n \in \mathbb{N}\}$ is finite, then there exists a value $x$ that appears infinitely many times in the sequence. So the subsequence constantly equal to $x$ converges to $x$.
\par Suppose the set of values $\{x_n: n \in \mathbb{N}\}$ is infinite. Since $X$ is first-countable, we can construct a countable basis $\{U_k\}$ at a limit point $a$ such that 
\begin{equation}
    x_{n_1} \in U_1, x_{n_2} \in U_2, \cdots, x_{n_k} \in U_k, \cdots
\end{equation}
and 
\begin{equation}
    U_1 \supseteq U_2 \supseteq \cdots \supseteq U_k \supseteq \cdots
\end{equation}
That is for each open set $U \ni a$ there exists $N$ such that $\forall k \ge N$, $U_k \subseteq U$. Then $x_{n_k} \in U_k \subseteq U$ for all $k \ge N$. So $x_{n_k} \to a$. \hfill $\blacksquare$


\par \noindent \textbf{Remark} For topological spaces, we have the following facts: 

\begin{enumerate}
    \item Compactness $\Rightarrow$ Limit Point Compactness
    \item Sequential Compactness $\Rightarrow$ Limit Point Compactness 
    \item But other implications are not true in general.
\end{enumerate}

\begin{theorem}
    Sequential compactness implies compactness for metric spaces.
\end{theorem}
\par \noindent \textbf{Proof} This is harder than other directions. 


\par \noindent \textbf{Exercise} Prove that $X$ is sequentially compact, then it is limit point compact. 

\section{Locally Compact} 

\begin{definition}
    A space $X$ is said to be locally compact at $x$ if there is some compact subspace $C$ of $X$ containing a neighborhood $U$ of $x$. If $X$ is locally compact at each of its points, then $X$ is said to be locally compact.
\end{definition}

\begin{example}
    $\mathbb{R}$ is locally compact. Because for any $x\in \mathbb{R}$, take $U = (x - 1, x + 1)$ and $C = [x - 1, x + 1]$ which is compact. 
\end{example}

\begin{example}
    $\mathbb{R}^n$ is locally compact. 
\end{example}

\begin{example}
    [non example] 
    $\mathbb{Q} \subseteq \mathbb{R}$ is not locally compact.
\end{example}

\par \noindent \textbf{Exercise} Show the above example. 
\par \noindent \textbf{Solution} Let $C$ be a compact subspace of $\mathbb{Q}$ containing a neighborhood $U$ of $x \in \mathbb{Q}$. Since $U$ is open in $\mathbb{Q}$, there exists an interval $(a, b) \subseteq \mathbb{R}$ such that $U = (a, b) \cap \mathbb{Q}$. We know $Q$ is dense in $\mathbb{R}$, so there exists an irrational number $y \in (a, b)$. Let $\{q_n\} \subseteq (a, b) \cap \mathbb{Q}$ be a sequence converging to $y$. Since $C$ is compact, so $\{q_n\}$ has a subsequence converging to a point in $C$. But a convergent sequence in $\mathbb{R}$ has a unique limit, so the subsequence converges to $y$, which is not in $\mathbb{Q}$ thus not in $C$. This is a contradiction. Hence, $\mathbb{Q}$ is not locally compact.
\mbox{} \\ \null \hfill $\blacksquare$ 

\begin{example}
    [non example] 
    $\mathbb{R}^{\omega}$ with product topology is not locally compact. If $U$ is open, then $U$ contains a basis element $(a_1, b_1) \times \cdots \times (a_n, b_n) \times \mathbb{R} \times \mathbb{R} \times \cdots$. Then $U$ is not contained in any compact set $C$($C$ contains one factor of $\mathbb{R}$, then the projection to some factor space is $\mathbb{R}$, but a projection is continuous, which means $\mathbb{R}$ is compact. This is definitely not true). 
\end{example}



\end{document}