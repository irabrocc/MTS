% 声明为子文件,指定主文件
\documentclass[main.tex]{subfiles}

\begin{document}
\pagestyle{plain}
\setcounter{chapter}{3}

\chapter{Later}
\label{chap:chapter4}

\begin{definition}
    A continuous bijective map $f: X \rightarrow Y$ is called a homeomorphism if $f^{-1} $ is also continuous. 
\end{definition}
\par \noindent \textbf{Remark} "homeomorphism" is not the same as "homomorphism".  
\begin{definition}
    $X \simeq Y$ are homeomorphic or topological equivalent if there exists a homeomorphism $f: X \rightarrow Y$
\end{definition}

\begin{definition}
    Two metric spaces $(X,d_X)$ and $(Y,d_Y)$ are called isometric if there exists a bijective map $f: X \rightarrow Y$ such that for all $x_1, x_2 \in X$, $d_Y(f(x_1), f(x_2)) = d_X(x_1, x_2)$. ($f$ is called an isometry)
\end{definition}
\par \noindent \textbf{Exercise} Prove that every isometry is a homeomorphism, but some homeomorphisms are not isometries. 
\begin{example}
    Let $D = \{(x,y) \in \mathbb{R}^2 | x^2 + y^2 \leq 1\}$ be the unit disk in $\mathbb{R}^2$ with the Euclidean metric. Let $S^1 = \{(x,y) \in \mathbb{R}^2 | \max{|x|, |y|} = 1\}$ be the unit square in $\mathbb{R}^2$ with the Euclidean metric. Then $D$ and $S^1$ are homeomorphic but not isometric.
\end{example}
\par \noindent \textbf{Proof}
\par The idea of prove homeomorphism is to map every radius of the disk to the corresponding line segment of the square. 

\par Assume there exists an isometry $f: D \rightarrow S^1$. Note that the diameter of $D$ is $2$, while the diameter of $S^1$ is $2\sqrt{2}$. This contradicts the definition of isometry. Thus, no such isometry exists. 

\begin{example}
    Another example of homeomorphism is the linear map from the interval $(0, 1)$ to $(a, b)$ (by stretching and compressing).
\end{example}

\begin{example}
    $(0, 1) \simeq (-\dfrac{\pi}{2}, \dfrac{\pi}{2})\simeq \mathbb{R}$ by the tangent function. 
\end{example}
\par The above example can be described graphically by a bowl(a circle centered at $(0, \frac{1}{2})$ with radius $\frac{1}{2}$ without the upper half) with radius $\dfrac{1}{2}$ on the real line.


\begin{property}
    If $f: X \rightarrow Y$ and $g: Y \rightarrow Z$ are continuous, then $g \circ f: X \rightarrow Z$ is continuous.
\end{property}

\begin{corollary}
    If $f: X \rightarrow Y$ and $g: Y \rightarrow Z$ are homeomorphisms, then $g \circ f: X \rightarrow Z$ is a homeomorphism. 
\end{corollary}

\begin{property}
    Homeomorphism $\simeq $ is an equivalence relation. 
\end{property}

\begin{definition}
    $X$ is said to be path-connected if any two points $x,y \in X$ can be joined by a path: there exists a continuous map $f: [0,1] \rightarrow X$ such that $f(0) = x$ and $f(1) = y$.
\end{definition}

\begin{definition}
    Let $X$ be a metric space. $X$ is said to be connected if one of the following equivalent conditions holds: 
    \begin{enumerate}
        \item $X$ cannot be represetned as $X = U_1 \sqcup U_2$ where $U_1, U_2$ are non-empty open subsets of $X$.
        \item $X$ cannot be represented as $X = V_1 \sqcup V_2$ where $V_1, V_2$ are non-empty closed subsets of $X$. 
        \item There is no proper non-empty subset $U \subseteq X$ which is both open and closed in $X$.
    \end{enumerate}
\end{definition}

\begin{property}
    If $X$ is path-connected and $f: X \rightarrow Y$ is a homeomorphism, then $Y$ is also path-connected. 
\end{property}
\par \noindent \textbf{Proof} Let $y_1, y_2 \in Y$. Since $f$ is bijective, there exist $x_1, x_2 \in X$ such that $f(x_1) = y_1$ and $f(x_2) = y_2$. Since $X$ is path-connected, there exists a continuous map $g: [0,1] \rightarrow X$ such that $g(0) = x_1$ and $g(1) = x_2$. Consider the map $h = f \circ g: [0,1] \rightarrow Y$. Since both $f$ and $g$ are continuous, $h$ is continuous. Moreover, $h(0) = f(g(0)) = f(x_1) = y_1$ and $h(1) = f(g(1)) = f(x_2) = y_2$. Thus, there exists a continuous path in $Y$ connecting $y_1$ and $y_2$, proving that $Y$ is path-connected. 
\\ \null \hfill $\blacksquare$

\begin{theorem}
    [Intermediate Value Theorem] Let $f: [a,b] \rightarrow \mathbb{R}$ be continuous. For any $c$ between $f(a)$ and $f(b)$, there exists some $x \in [a,b]$ such that $f(x) = c$. 
\end{theorem}

\begin{example}
    $\mathbb{R}\setminus \{0\}$ is not path-connected. 
\end{example}
\par \noindent \textbf{Proof} By IVT. 

\begin{theorem}
    [Intermediate Value Theorem for Path-Connectedness] Let $X$ be a path-connected space and $f: X \rightarrow \mathbb{R}$ be continuous. Suppose there exist $x_1, x_2 \in X$ such that $f(x_1) = c_1$ and $f(x_2) = c_2$. Then for any $c$ between $c_1$ and $c_2$, there exists some $x \in X$ such that $f(x) = c$.
\end{theorem}
\par \noindent \textbf{Proof} Since $X$ is path-connected, there exists a continuous map $g: [0,1] \rightarrow X$ such that $g(0) = x_1$ and $g(1) = x_2$. Consider the map $h = f \circ g: [0,1] \rightarrow \mathbb{R}$. Since both $f$ and $g$ are continuous, $h$ is continuous. Moreover, $h(0) = f(g(0)) = f(x_1) = c_1$ and $h(1) = f(g(1)) = f(x_2) = c_2$. By the Intermediate Value Theorem, for any $c$ between $c_1$ and $c_2$, there exists some $t \in [0,1]$ such that $h(t) = c$. Let $x = g(t)$. Then $f(x) = f(g(t)) = h(t) = c$. Thus, there exists some $x \in X$ such that $f(x) = c$. 
\\ \null \hfill $\blacksquare$

\begin{property}
    Let $X$ be a metric space. Then IVT for $X$ holds if and only if there is no continuous and surjective map $f: X \rightarrow \{0, 1\}$. 
\end{property}
\par \noindent \textbf{Proof} 
\par We assume that $X$ contains more than one point. 
\par One direction is trivial. 
\par Let's assume that the IVT doesn't hold for $X$.
\par Since the IVT does not hold for $X$, there exists $y_1, y_2\in \mathbb{R}$ such that there exists $y_3$ between $y_1$ and $y_2$ which is not in the range of $f$. Then $f: X \rightarrow \mathbb{R}\setminus \{y_3\}$. Now we can easily define a continuous and surjective map $\tilde{f}: X \rightarrow \{0, 1\}$. 
\mbox{} \\ \null \hfill $\blacksquare$ 

\begin{property}
    Let $X$ be a metric space. $X$ is connected if and only if IVT for $X$ holds.
\end{property}
\par \noindent \textbf{Proof} 
\par The statement is equivalent to: $X$ is connected if and only if there is no continuous and surjective map $f: X \rightarrow \{0, 1\}$. 
\par Suppose there is a continuous and surjective map $f: X \rightarrow \{0, 1\}$. Then $f^{-1}(\{0\})$ and $f^{-1}(\{1\})$ are non-empty, disjoint, open subsets of $X$ whose union is $X$. Thus, $X$ is not connected. 
\par Suppose that there is no continuous and surjective map $f: X \rightarrow \{0, 1\}$. If $X$ is not connected, then there exist non-empty, disjoint, open subsets $U_1, U_2$ of $X$ such that $X = U_1 \sqcup U_2$. We can define a map $f: X \rightarrow \{0, 1\}$ by setting $f(x) = 0$ if $x \in U_1$ and $f(x) = 1$ if $x \in U_2$. This map is continuous and surjective, contradicting our assumption. Therefore, $X$ must be connected.
\mbox{} \\ \null \hfill $\blacksquare$ 

\begin{corollary}
    Let $X$ be a metric space. If $X$ is path-connected, then $X$ is connected. 
\end{corollary} 
\par \noindent \textbf{Proof} Since $X$ is path-connected, IVT for $X$ holds. Thus, $X$ is connected.
\mbox{} \\ \null \hfill $\blacksquare$ 

\par \noindent \textbf{Remark} IVT holds for $X$ doesn't implies $X$ is path-connected. 
\begin{example}
    This is an example of a connected space but not path-connected:
    \begin{equation}
        X = \{(x, \sin{\frac{1}{x}}) | x > 0\} \cup (\{0\}\times [-1, 1])
    \end{equation}
\end{example}



\end{document}