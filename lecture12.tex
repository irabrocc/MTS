% 声明为子文件,指定主文件
\documentclass[main.tex]{subfiles}

\begin{document}
\pagestyle{plain}
\setcounter{chapter}{11}

\chapter{Later}
\label{chap:chapter1}

\begin{definition}
    A collection $\mathcal{A}$ of subsets of a space $X$ is called "a covev" or "a covering" if the union of the elements of $\mathcal{A}$ is $X$; it is called an open covering if all these sets are open. 
\end{definition}

\begin{definition}
    $X$ is called a compact space if every open covering $\mathcal{A}$ contains a finite subcollection which also covers $X$. 
\end{definition}
\par \noindent \textbf{Remark} Being compact depends on the topology: For example, consider $X$ with the antidiscrete topology, i.e., $\mathcal{T} = \{\emptyset, X\}$. Then $X$ is compact. 
\begin{example}
    $\mathbb{R}$ with the standard topology is not compact. Consider the open covering $\mathcal{A} = \{(-n, n) | n \in \mathbb{N}\}$. 
\end{example}
\begin{example}
    $(0, 1]$ with the standard topology is not compact. Consider the open covering $\mathcal{A} = \{(1/n, 1] | n \in \mathbb{N}\}$.
\end{example}
\begin{example}
    \begin{itemize}
        \item $\{0\}\cup \{\dfrac{1}{n}| n\in \mathbb{Z}^{+}\} \subseteq \mathbb{R}$ is compact because $\{0\}$ is the limit point and any open covering must contain an open set containing $0$, which covers all but finite points.
        \item $\{\dfrac{1}{n}| n\in \mathbb{Z}^+\}$ is not compact. Consider the open covering $\mathcal{A} = \{(1/n - \epsilon, 1/n + \epsilon) | n \in \mathbb{N}\}$. (Similarly, for infinite $X$ with discrete topology, $X$ is not compact.)
    \end{itemize}
\end{example}

\begin{definition}
    $Y \subseteq X$ is compact if every open covering of $Y$ by sets open in $X$ contains a finite subcollection covering $Y$.
\end{definition}
\begin{lemma}
    $Y \subseteq X$ is compact if and only if every open covering of $Y$ by sets open in $Y$ contains a finite subcollection covering $Y$.
\end{lemma}
\par \noindent \textbf{Proof}
\par Trivial. 

\begin{theorem}
    Every closed subspace of a compact space is compact. 
\end{theorem}
\par \noindent \textbf{Proof} Let $X$ be a compact space and let $Y$ be a closed subspace of $X$. Let $\mathcal{A}$ be an open covering of $Y$ by sets open in $X$. Since $Y$ is closed, $X\setminus Y$ is open in $X$. Thus, $\mathcal{A} \cup \{X\setminus Y\}$ is an open covering of $X$. By the compactness of $X$, there exists a finite subcollection $\mathcal{A}' \subseteq \mathcal{A}$ such that $\mathcal{A}' \cup \{X\setminus Y\}$ covers $X$. Therefore, $\mathcal{A}'$ covers $Y$. Hence, $Y$ is compact.
\begin{theorem}
    Every compact subspace of a Hausdorff space is closed. 
\end{theorem}
\par \noindent \textbf{Proof} 
\par It suffices to show that for any $x \in X \setminus Y$, there exists an open set $U$ containing $x$ such that $U \cap Y = \emptyset$. For each $y \in Y$, since $X$ is Hausdorff, there exist disjoint open sets $U_y$ and $V_y$ such that $x \in U_y$ and $y \in V_y$. Then $\{V_y | y \in Y\}$ is an open covering of $Y$. By the compactness of $Y$, there exists a finite subcollection $\{V_{y_1}, V_{y_2}, \ldots, V_{y_n}\}$ that covers $Y$. Let
\begin{equation}
    U = \bigcap_{i=1}^{n} U_{y_i}
\end{equation}
Then $U$ is an open set containing $x$. Moreover
\begin{equation}
    U \cap Y  = (\bigcap_{i=1}^{n} U_{y_i}) \cap Y = \emptyset 
\end{equation}
Thus, $U \cap Y = \emptyset$. Hence, $Y$ is closed. 
\mbox{} \\ \null \hfill $\blacksquare$ 
\begin{lemma}
    $Y$ is a compact subspace of Hausdorff space $X$. And $x \in X \setminus Y$. Then there exist open sets $U$ and $V$ such that $x \in U$, $Y \subseteq V$ and $U \cap V = \emptyset$.
\end{lemma}
\par \noindent \textbf{Proof} Proved inside the proof of the theorem. 

\begin{theorem}
    The image of a compact set under a continuous map is compact. 
\end{theorem}
\par \noindent \textbf{Proof} Let $f: X \rightarrow Y$ be a continuous map. Consider any open covering $\mathcal{A}$ of $f(X)$ by sets open in $Y$. Then $\{f^{-1}(U) | U \in \mathcal{A}\}$ is an open covering of $X$ by sets open in $X$. By the compactness of $X$, there exists a finite subcollection $\{f^{-1}(U_1), f^{-1}(U_2), \ldots, f^{-1}(U_n)\}$ that covers $X$. Thus, $\{U_1, U_2, \ldots, U_n\}$ is a finite subcollection of $\mathcal{A}$ that covers $f(X)$. Hence, $f(X)$ is compact.
\mbox{} \\ \null \hfill $\blacksquare$ 

\begin{theorem}
    Let $f: X \rightarrow Y$ be a bijective continuous map. If $X$ is compact and $Y$ is Hausdorff, then $f$ is a homeomorphism.
\end{theorem}
\par \noindent \textbf{Proof} 
\par We just combine the above theorems. 
\par It suffices to show that images of closed sets under $f$ are closed. Consider any closed set $D$ in $X$. By the theorem, $D$ is compact. Thus, by the previous theorem, $f(D)$ is compact. Since $Y$ is Hausdorff, by another theorem, $f(D)$ is closed. Hence, $f$ is a homeomorphism.
\mbox{} \\ \null \hfill $\blacksquare$  

\begin{lemma}
    [tube lemma]
    Let $x_0\in X$, suppose $\{x_0\}\times Y$ is convered by open sets $W_i$ in $X \times Y$. Then one can choose a finite subcovering $W_1, W_2, \ldots, W_n$ of $\{x_0\}\times Y$ and find an open neighbor $U$ of $x_0$ such that $U \times Y \subseteq \bigcup_{i=1}^{n} W_i$.
\end{lemma}
\par \noindent \textbf{Proof} One may assume that $W_i = U_i \times V_i$ where $U_i$ is open in $X$ and $V_i$ is open in $Y$. Then $\{V_i\}$ is an open covering of $Y$. By the compactness of $Y$, there exists a finite subcollection $\{V_1, V_2, \ldots, V_n\}$ that covers $Y$. 

\par To be done. 

\begin{theorem}
    The product of finitely many compact spaces is compact. 
\end{theorem}
\par \noindent \textbf{Proof} It is enough to prove for two sets. Let $X$ and $Y$ be compact spaces. 
\par We can cover $X\times Y$ by finitly many tubes by the tube lemma. And each tube can be covered by finitly many open sets. Thus $X \times Y$ is compact.
\par To be done. 

\begin{definition}
    A collection $\mathcal{C}$ of sets has the finite intersection property if for any finite subcollection $\{C_1, C_2, \ldots, C_n\} \subseteq \mathcal{C}$, we have
    \begin{equation}
        C_1 \cap C_2 \cap \ldots \cap C_n \neq \emptyset
    \end{equation}
\end{definition}

\begin{theorem}
    $X$ is a topological space. Then $X$ is compact if and only if for every collection $\mathcal{C}$ of closed sets in $X$ with the finite intersection property, we have
    \begin{equation}
        \bigcap_{C \in \mathcal{C}} C \neq \emptyset
    \end{equation}
\end{theorem}

\par \noindent \textbf{Proof} 
\par Let $\mathcal{A}$ be a collection of open sets in $X$. Then $\mathcal{C} = \{X \setminus A | A \in \mathcal{A}\}$ is a collection of closed sets in $X$. 
\par Then $\mathcal{A}$ is an open covering of $X$ if and only if $\bigcap_{C \in \mathcal{C}} C = \emptyset$. 
\par Then a finite subcollection of $\mathcal{A}$ covers $X$ if and only if the corresponding finite subcollection of $\mathcal{C}$ has empty intersection. 
\par Now the theorem follows directly.
\mbox{} \\ \null \hfill $\blacksquare$ 

\end{document}