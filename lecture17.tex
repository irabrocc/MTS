% 声明为子文件,指定主文件
\documentclass[main.tex]{subfiles}

\begin{document}
\pagestyle{plain}
\setcounter{chapter}{16}

\chapter{Later}
\label{chap:chapter17}

\begin{theorem}\label{thm:17.0.1}
    Let $X$ be a topological space. Then $X$ is Hausdorff locally compact if and only if there exists a space $Y$ such that
    \begin{enumerate}
        \item $X$ is a subspace of $Y$.
        \item $Y \setminus X$ contains exactly one point $p$.
        \item $Y$ is compact Hausdorff. 
    \end{enumerate}
    This $Y$ is unique in the following sense: If $Y, Y'$ are two spaces with the above properties and $Y = X \cup \{p\}, Y' = X \cup \{q\}$, then there exists a homeomorphism $h: Y \to Y'$ such that $h(x) = x$ for all $x \in X$ and $h(p) = q$.
\end{theorem}
\par \noindent \textbf{Proof} 
\par \noindent \textbf{Uniqueness} Let $Y = X \cup \{p\}$, $Y' = X \cup \{q\}$ satisfies the above properties. Define $h: Y \to Y'$ as follows: $h(x) = x$ for all $x \in X$ and $h(p) = q$. We show that $h$ is continuous. But the function is symmetric, so it is enough to show that $h(U)$ is open in $Y'$ for all open set $U$ in $Y$. 
\par Take $U$ be open. There are two cases. 
\par If $U \subseteq X$, then we are done. 
\par Suppose $p\in U$. Then $C = Y \setminus U$ is closed in $Y$. So $C$ is compact. 
\par \noindent \textbf{Construction} We introduce the topology on $Y = X \cup \{\infty\}$ as follows: 
\par There are two types of open sets in $Y$: 
\begin{enumerate}
    \item $U \subseteq X \subseteq Y$ is open in $X$.
    \item $U = Y \setminus C$ where $C$ is compact subspace of $X$. 
\end{enumerate}

We need to check that this is a topology. 
\par For the intersection of two open sets of type 1, we have $U_1 \cap U_2$ is open in $X$ thus is open in $Y$. 
\par For the intersection of type 1 and type 2, we have $U_1 \cap (Y \setminus C) = U_1\cap(X \setminus C)$ which is the union of open sets in $X$ thus is open in $Y$ because $C$ is closed in $X$( as a compact subspace of a Hausdorff space). 
\par For the intersection of two open sets of type $2$, we have $(Y \setminus C_1) \cap (Y \setminus C_2) = Y \setminus (C_1 \cup C_2)$ which is open in $Y$ because $C_1 \cup C_2$ is compact. 
\par Similarly, one checks that the union of any collection of open sets is open. We have 
\begin{enumerate}
    \item $\bigcup U_{\alpha} = U$ is of type 1. 
    \item $\bigcup (Y \setminus C_{\alpha}) = Y \setminus \bigcap C_{\alpha} = Y \setminus C$ where $C = \bigcap C_{\alpha}$ is compact(Easy to check). This is of type 2. 
    \item $\bigcup U_{\alpha} \cup \bigcup (Y \setminus C_{\beta}) = U \cup (Y \setminus C) = Y \setminus (C \setminus U)$. This is of type 2 because $C \setminus U = C \cap (X \setminus U)$ is closed in $C$ thus is compact because a closed subset of a compact set is compact.
\end{enumerate}
\par Now we show that $X$ is a subspace of $Y$. Given any open set $V$ of $Y$, we show its intersection with $X$ is open in $X$. If $V$ is of type 1, then $V \cap X = V$ which is open in $X$. If $V$ is of type 2, then $V \cap X = (Y \setminus C) \cap X = X \setminus C$ which is open in $X$ because $C$ is closed in $X$ as a compact subspace of a Hausdorff space. Conversely, given any open set $U$ of $X$, $U$ is an open set of $Y$ of type 1. So $X$ is a subspace of $Y$. 

\par We show that $Y$ is compact. 
\par Let $\mathcal{A}$ be an open covering of $Y$. The collection $\mathcal{A}$ must contain an open set containing $\infty$. So there exists a compact set $C$ such that $U = Y \setminus C$ with $U \in \mathcal{A}$. Take all the members of $\mathcal{A}$ different from $U$ and intersect them with $X$. This gives a collection of open sets in $X$ which covers $C$. Since $C$ is compact, there exists a finite subcovering of $C$, say $\{V_1, V_2, \cdots, V_n\}$. Then $\{U, V_1, V_2, \cdots, V_n\}$ is a finite subcovering of $Y$. So $Y$ is compact.  
\par We show that $Y$ is Hausdorff.
\par Take two distinct points $x, y \in Y$. There are two cases.
\begin{enumerate}
    \item If $x, y \in X$, since $X$ is Hausdorff, there exist disjoint open sets $U, V$ in $X$ such that $x \in U, y \in V$. Then $U, V$ are open in $Y$ and disjoint.
    \item If one of them is $\infty$, say $y = \infty$. Since $X$ is locally compact at $x$, there exists an open neighborhood $U$ of $x$ and a compact set $C$ such that $U \subseteq C$. So $C$ is closed in $X$ thus in $Y$. Then $V = Y \setminus C$ is an open neighborhood of $\infty$. Clearly, $U \cap V = \emptyset$.
\end{enumerate}
\par Finally we prove the other direction. Suppose a space $Y$ satisfying conditions (1)-(3) exists. Then $X$ is Hausdorff as a subspace of a Hausdorff space. Given any $x \in X$, we show $X$ is locally compact at $x$. Choose disjoint open sets $U, V$ in $Y$ such that $x \in U$ and $p \in V$. Then $C = Y \setminus V$ is compact as a closed subset of a compact space. Also, $x \in U \subseteq C$. So $X$ is locally compact at $x$. 

\mbox{} \\ \null \hfill $\blacksquare$ 

\begin{definition}
    If $Y$ and $X$ are as in the above theorem, then $Y$ is called the one-point compactification of $X$. 
\end{definition}

\begin{example}
    \begin{enumerate}
        \item The one point compactification of $\mathbb{R}$ is homeomorphic to $S^1$. We denote $\overline{\mathbb{R}} \cong S^1$.
        \item The one point compactification of $\mathbb{R}^n$ is homeomorphic to $S^n$. We denote $\overline{\mathbb{R}^n} \cong S^n$. 
        \item The one point compactification of $\overline{\mathbb{C}} = \mathbb{C} \cup \{\infty\}$(Riemann sphere) is homeomorphic to $S^2$. 
    \end{enumerate}
\end{example}


\begin{theorem}
    Let $X$ be a Hausdorff space. Then $X$ is locally compact if and only if for any $x\in X$ and any neighborhood $U$ of $x$, there exists a neighborhood $V$ of $x$ such that $\bar{V}$ is compact and $\bar{V} \subseteq U$.
\end{theorem}

\par \noindent \textbf{Proof} 
\par $\Leftarrow$ $x\in V \subseteq \bar{V} \subseteq U$ and $\bar{V}$ is compact. So $X$ is locally compact at $x$.
\par $\Rightarrow$ Let $Y$ be the one-point compactification(compact and Hausdorff) of $X$. Let $U$ be a neighborhood of $x$ in $Y$. Then $Y\setminus U:= C$ is a closed subset of $Y$. So $C$ is compact subspace of $Y$(Hausdorff). By lemma 12.0.2, there exists two open sets $V, W$ in $Y$ such that $x \in V$, $C \subseteq W$ and $V \cap W = \emptyset$. Then $\bar{V}$ is compact and $\bar{V} \cap C = \emptyset$. So $\bar{V} \subseteq U$. \hfill $\blacksquare$

\begin{corollary}\label{coro:17.0.1}
    Let $X$ be locally compact Hausdorff space. $A \subseteq X$ is open or closed. Then $A$ is locally compact.
\end{corollary}

\par \noindent \textbf{Proof} 
\par Let $A \subseteq X$ be closed. Given $x\in A$, since $X$ is locally compact, there exists an open neighborhood $U$ of $x$ in $X$ and a compact set $C$ such that $x \in U \subseteq C$. Then $C \cap A$ is closed in $C$ thus is compact, and it contains the neighborhood $U \cap A$ of $x$ in $A$. So $A$ is locally compact at $x$. 

\par Let $A \subseteq X$ be open. Given $x\in A$, since $X$ is locally compact and $X$ is Hausdorff, by the previous theorem, there exists a neighborhood $V$ of $x$ such that $\bar{V}$ is compact and $\bar{V} \subseteq A$. Then $C = \bar{V}$ is a compact set containing the neighborhood $V$ of $x$ in $A$. So $A$ is locally compact at $x$. 
\mbox{} \\ \null \hfill $\blacksquare$ 

\begin{corollary}
    $X$ is locally compact Hausdorff if and only if $X$ is homeomorphic to an open subspace of a compact Hausdorff space. 
\end{corollary}

\par \noindent \textbf{Exercise} Show the above corollary by \ref{thm:17.0.1} and \ref{coro:17.0.1}.
\par \noindent \textbf{Solution} 
\par $\Rightarrow$ Let $Y$ be the one-point compactification of $X$. Recall that Hausdorff is $T_1$. So $X = Y \setminus \{p\}$ is open in $Y$. The identity map $id: X \to X$ is a homeomorphism from $X$ to the open subspace $X$ of $Y$. 
\par $\Leftarrow$ Let $X$ be homeomorphic to an open subspace $U$ of a compact Hausdorff space $Z$. Since $Z$ is compact Hausdorff, $Z$ is locally compact Hausdorff. By \ref{coro:17.0.1}, $U$ is locally compact Hausdorff. Since $X$ is homeomorphic to $U$, $X$ is locally compact Hausdorff.
\mbox{} \\ \null \hfill $\blacksquare$ 

\section{Urysohn's Metrization Theorem}
\begin{theorem}
    Every $X$ that is regular(T3) and second-countable is metrizable.
\end{theorem}



\end{document}