% 声明为子文件,指定主文件
\documentclass[main.tex]{subfiles}

\begin{document}
\pagestyle{plain}
\setcounter{chapter}{16}

\chapter{Later}
\label{chap:chapter17}

\begin{theorem}
    Let $X$ be second countable. Then: 
    \begin{enumerate}
        \item Every covering of $X$ by open sets has a countable subcovering($X$ is Lindelöf space). 
        \item There exists a countable dense subset $A \subseteq X$ such that $\bar{A} = X$. (In this case $X$ is called separable.)
    \end{enumerate}
\end{theorem}

\begin{property}
    A second countable space is: 
    \begin{enumerate}
        \item First countable.
        \item Separable.
        \item Lindelöf.
    \end{enumerate}
\end{property}

\begin{example}
    [non-example] 
    $\mathbb{R}_l$ with lower limit topology(Sorgengrey line) is first countable, Lindelöf, seperable, but not second countable. 
    \par First-countable: for any $x \in \mathbb{R}_l$, $\{[x, x + \frac{1}{n}): n \in \mathbb{N}\}$ is a countable local basis at $x$. 
    \par Assume the contrary that $\mathbb{R}_l$ is second countable. Let $\mathcal{B} = \{B_n: n \in \mathbb{N}\}$ be a countable basis for $\mathbb{R}_l$. For each $x \in \mathbb{R}$, there exists $B_{n_x} \in \mathcal{B}$ such that $x \in B_{n_x} \subseteq [x, x + 1)$. So $\min B_{n_x} = x$. This implies that the map $f: \mathbb{R} \to \mathcal{B}, f(x) = B_{n_x}$ is injective. However, $\mathbb{R}$ is uncountable while $\mathcal{B}$ is countable, which is a contradiction. So $\mathbb{R}_l$ is not second countable. 
    \par Separable: $\mathbb{Q} \subseteq \mathbb{R}_l$ is countalbe and dense in $\mathbb{R}_l$. 
    \par Lindelöf: See the below property. 
\end{example}

\begin{property}
    [Lindelöf property of $\mathbb{R}_l$] 
    Any open cover of $\mathbb{R}_l$ has a countable subcover.
\end{property} 

\par \noindent \textbf{Proof} 
\par Let $\mathcal{A}$ be a covering of $\mathbb{R}_l$ by $[a_{\alpha}, b_{\alpha})$'s. We need to show that there exists a countable subcovering of $\mathcal{A}$ covering $\mathbb{R}_l$. 
\par Let $C = \bigcup(a_{\alpha}, b_{\alpha})$ such that $\mathbb{R}\setminus C$ is not empty. Let $x = a_{\beta}$ for some $\beta$ such that $x\notin (a_{\alpha}, b_{\alpha})$. Take $q_x \in (a_{\beta}, b_{\beta}) \cap \mathbb{Q}$ thus $x < q_x$. 
\par Take $x, y\in \mathbb{R}$. Then $q_x < q_y$ or we make a contradiction(why?). The map $x \mapsto q_x$ is injective from $\mathbb{R}\setminus C$ to $\mathbb{Q}$. However

\begin{property}
    $\mathbb{R}_l \times \mathbb{R}_l$ is not Lindelöf. 
\end{property}
\par \noindent \textbf{Proof} 


\begin{example}
    Subspace of Lindelöf space is not necessarily Lindelöf.
\end{example}

\section{Separation Axioms}
\par Topological space $X$ can satisfy the following separability axioms:
\begin{enumerate}
    \item $T1$(Frechet): Given two distinct points $x, y \in X$, there exists an open set $U$ such that $x \in U$ and $y \notin U$. (Equivalently, all singleton sets are closed. Easy exercise)
    \item $T2$(Hausdorff): Given two distinct points $x, y \in X$, there exist open sets $U, V$ such that $x \in U, y \in V$ and $U \cap V = \emptyset$.
    \item $T3$(Regular): $X$ is $T1$ and given a closed set $F \subseteq X$ and a point $x \notin F$, there exist open sets $U, V$ such that $x \in U, F \subseteq V$ and $U \cap V = \emptyset$.
    \item $T4$(Normal): $X$ is $T1$ and given two disjoint closed sets $F_1, F_2 \subseteq X$, there exist open sets $U, V$ such that $F_1 \subseteq U, F_2 \subseteq V$ and $U \cap V = \emptyset$.
\end{enumerate}

\begin{lemma}
    Let $X$ be a $T1$ topological space. Then
    \begin{enumerate}
        \item $X$ is regular if and only if for every point $x \in X$ and every neighborhood $U$ of $x$, there exists a neighborhood $V$ of $x$ such that $\bar{V} \subseteq U$. 
        \item $X$ is normal if and only if for every closed set $F \subseteq X$ and every neighborhood $U$ of $F$, there exists a neighborhood $V$ of $F$ such that $\bar{V} \subseteq U$.
    \end{enumerate}
\end{lemma}

\par \noindent \textbf{Proof} 
\par $\Rightarrow$ Let $X$ be regular. 
\par If $U = X$, there is nothing to prove.
\par Suppose $U \neq X$. Then $F = X \setminus U$ is closed and $x \notin F$. By regularity of $X$, there exist open sets $V, W$ such that $x \in V, F \subseteq W$ and $V \cap W = \emptyset$. Now we need to show $\bar{V} \subseteq U$. 
\par If $y\in F$, then $y \in W$. Since $V \cap W = \emptyset$, $y \notin \bar{V}$ because we find a neighborhood $W$ of $y$ such that $W \cap V = \emptyset$. Thus, $\bar{V} \subseteq X \setminus F = U$.
\par $\Leftarrow$ To be done. 

\par The proof of 2. is similar. 

\mbox{} \\ \null \hfill $\blacksquare$ 

\begin{theorem}
    \begin{enumerate}
        \item $X$ is Hausdorff, then $Y \subseteq X$ with subspace topology is Hausdorff.
        \item $X_{\alpha}$ is Hausdorff, then $\prod X_{\alpha}$ with product topology is Hausdorff.
        \item $X$ is regular, then $Y \subseteq X$ with subspace topology is regular. 
        \item $X_{\alpha}$ is regular, then $\prod X_{\alpha}$ with product topology is regular. 
    \end{enumerate}
\end{theorem}
\par \noindent \textbf{Proof}
\par (a) is obvious. 
\par (b) Let $x = (x_{\alpha}), y = (y_{\alpha}) \in \prod X_{\alpha}$ with $x \neq y$. Then there exists $\beta$ such that $x_{\beta} \neq y_{\beta}$. Since $X_{\beta}$ is Hausdorff, there exist open sets $U_{\beta}, V_{\beta} \subseteq X_{\beta}$ such that $x_{\beta} \in U_{\beta}, y_{\beta} \in V_{\beta}$ and $U_{\beta} \cap V_{\beta} = \emptyset$. 

\par (c) is obvious. 

\par (d) To be done. 


\par \noindent \textbf{Remark} $X$ is normal does not imply that $Y \subseteq X$ with subspace topology is normal. $X_1, X_2$ are normal does not imply that $X_1 \times X_2$ with product topology is normal. 

\par \noindent \textbf{Exercise} Show that $\mathbb{R}_l$ is normal. 

\par \noindent \textbf{Fact} (Without Proof) $\mathbb{R}_l^2$ is not normal. But it is regular( as a product of regular spaces). 
\par \noindent \textbf{Exercise} $\mathbb{R}_K$ is a topological space with basis given by all open intervals $(a, b)$ and all sets of the form $(a, b) \setminus K$ where $K = \{1/n | n \in \mathbb{N}\}$. $\mathbb{R}_K$ is Hausdorff. But $\mathbb{R}_K$ is not regular(To be done).  


\end{document}