% 声明为子文件,指定主文件
\documentclass[main.tex]{subfiles}

\begin{document}
\pagestyle{plain}
\setcounter{chapter}{6}

\chapter{Lecture7}
\label{chap:Lecture7}

\section{Subspace Topology}
\begin{definition}
    Let \((X, \mathcal{T})\) be a topological space and \(Y \subseteq X\). The \textbf{subspace topology} on \(Y\) is defined as
    \[
        \mathcal{T}_Y = \{ U \cap Y : U \in \mathcal{T} \}.
    \]
\end{definition}
\begin{property}
    $(Y, \mathcal{T}_Y)$ is a topological space.
\end{property}
\begin{lemma}
    If $\mathcal{B}$ is a basis for the topology $\mathcal{T}$ on $X$, then the collection
    \[
        \mathcal{B}_Y = \{ B \cap Y : B \in \mathcal{B} \}
    \]
    is a basis for the subspace topology $\mathcal{T}_Y$ on $Y$.
\end{lemma}

\par \noindent \textbf{Proof} 
\par Let $U \in \mathcal{T}$ and $y\in U \cap Y$. Then there exists $B \in \mathcal{B}$ such that $y \in B \subseteq U$. Then $y \in B \cap Y \subseteq U \cap Y$. 

\par \noindent \textbf{Remark} The set which is open in the subspace topology may not be open in the original topology.

\begin{property}
    If $Y$ is open in $(X, \mathcal{T})$ and $U \subseteq Y$ is open in $(Y, \mathcal{T}_Y)$, then $U$ is open in $(X, \mathcal{T})$. 
\end{property}
\par \noindent \textbf{Proof} Since $U \subseteq Y$ is open in $(Y, \mathcal{T}_Y)$, there exists $V \in \mathcal{T}$ such that $U = V \cap Y$. Since $Y$ is open in $(X, \mathcal{T})$, we have $U = V \cap Y$ is open in $(X, \mathcal{T})$.

\begin{theorem}
    Let $A \subseteq X$ and $B \subseteq Y$ such that $A \times B \subseteq X \times Y$. Then the product of subspace topologies $\mathcal{T}_{A} \times \mathcal{T}_{B}$ is equal to the subspace topology $\mathcal{T}_{A \times B}$ on $A \times B$. 
\end{theorem}
\par \noindent \textbf{Proof} 
\par Let $U \subseteq X$ and $V \subseteq Y$ be open sets in $(X, \mathcal{T}_X)$ and $(Y, \mathcal{T}_Y)$ respectively. Then the products of the form $U \times V$ form a basis for the product topology on $X \times Y$. And we have 
\begin{equation}
    (U \times V) \cap (A \times B) 
\end{equation}
form a basis for the subspace topology on $A \times B$. Note that
\begin{equation}
    (U \times V) \cap (A \times B) = (U \cap A) \times (V \cap B)
\end{equation} 
So the basis for the product of subspace topologies is equal to the basis for the subspace topology on $A \times B$. 
\\ \null \hfill $\blacksquare$ 
\par Let $(X, <)$ be an ordered set. Let $a_0 = min(X)$ and $b_0 = max(X)$ if they exists. Then 
\begin{equation}
    \mathcal{B} = \{ (a, b) : a_0 < a < b < b_0 \} \cup \{ [a_0, b) : b < b_0 \} \cup \{ (a, b_0] : a > a_0 \}
\end{equation}
is a basis for the order topology on $X$. Notice that $[a_0, b)$ refers to the set $\{ x \in X : x < b \}$ when $a_0$ does not exist. 
\par \noindent \textbf{Remark} Let $Y \subseteq X$ be a subset inheriting the order from $X$. Then it may happen that the order topology on $Y$ is different from the subspace topology on $Y$.
\begin{example}
    Let $X = \mathbb{R}$ with the usual order and $Y = [0, 1)\cup \{2\}$ be a subset of $X$. Then $\{2\}$ is open in the subspace topology on $Y$ since \(\{2\} = (1.5, 2.5) \cap Y\). However, $\{2\}$ is not open in the order topology on $Y$. 
\end{example}

\par \noindent \textbf{Exercise} Show that for the order topology $[0,1) \cup \{2\}$ is connected. 

\begin{example}
    Let $X = Y = \mathbb{R}$ with the usual order. Then the product of order topologies is the standard topology on \(\mathbb{R}^2\). We take the lexicographic order as the order on \(\mathbb{R}^2\).  
\end{example}
\par \noindent \textbf{Remark} Let $I\times I = [0, 1]\times [0, 1] \subseteq X \times Y$. Then the order topology on $[0, 1] \times [0, 1]$ is different from the subspace topology inherited from the order topology on $\mathbb{R}^2$. The set $\{1/2\} \times (1/2, 1]$ is not open in the order topology on $I \times I$ but it is open in the subspace topology inherited from $\mathbb{R}^2$.
\begin{definition}
    $Y$ is convex if for any $a, b \in Y$, the interval $(a, b) \subseteq Y$. 
\end{definition}
\begin{theorem}
    Let $Y \subseteq X$ be a convex subset of $(X, <)$. Then the the restriction of the order topology on $X$ to $Y$ is equal to the order topology on $Y$.
\end{theorem}

\par \noindent \textbf{Proof} 
\par Take $(a, +\infty), (-\infty, b) \subseteq X$ which form a basis for the order topology on $X$. Take $Y \subseteq X$ convex. 
\begin{enumerate}
    \item If $a\in Y$, then \((a, +\infty) \cap Y = \{y| y\in Y \text{ and } a < y\}\).  
    \item If $a \notin Y$, then there are two cases:
    \begin{enumerate}
        \item If $a$ is a lower bound for $Y$, then \((a, +\infty) \cap Y = Y\)
        \item If $a$ is an upper bound for $Y$, then $(a, +\infty) \cap Y = \emptyset$
    \end{enumerate} 
\end{enumerate}
Then we obtain the basis for subspace topology on $Y$ from the basis of order topology on $X$. 
And also this is the basis for the order topology on $Y$. 

\begin{definition}
    $A \subseteq X$ is closed if and only if \(X \setminus A\) is open. 
\end{definition}
\begin{property}
    \begin{enumerate}
        \item $X$ and $\emptyset$ are closed.
        \item The intersection of any collection of closed sets is closed.
        \item The union of finitely many closed sets is closed.
    \end{enumerate}
\end{property}
\begin{theorem}
    Let $X$ be a topological space. And $Y \subseteq X$ has a subspace topology. Then $A \subseteq Y$ is closed in $Y$ if and only if there exists a closed set $C$ in $X$ such that \(A = C \cap Y\).
\end{theorem}
\par \noindent \textbf{Proof} 
\par \((\Rightarrow)\) Since $A$ is closed in $Y$, then \(Y \setminus A\) is open in $Y$. So there exists an open set $U$ in $X$ such that \(Y \setminus A = U \cap Y\). Let \(C = X \setminus U\). Then $C$ is closed in $X$ and
\[
    A = Y \setminus (Y \setminus A) = Y \setminus (U \cap Y) = Y \cap (X \setminus U) = Y \cap C.
\]
\par \((\Leftarrow)\) Let $A = C \cap Y$ where $C$ is closed in $X$. Then \(X \setminus C\) is open in $X$. So we have
$ (X \setminus C) \cap Y$ is open in $Y$. But we have 
\begin{equation}
    (X \setminus C) \cap Y = Y \setminus (C \cap Y) = Y \setminus A.
\end{equation}
which is open in $Y$. So $A$ is closed in $Y$.

\begin{definition}
    If $U \subseteq X$ is open and $x \in U$ then $U$ is a neighborhood of $x$.
\end{definition}
\begin{definition}
    $Int(A) = \bigcup \{U \subseteq X | U \text{ is open and } U \subseteq A\}$ is the interior of $A$.
    $\bar{A} = \bigcap \{C \subseteq X | C \text{ is closed and } A \subseteq C\}$ is the closure of $A$.
\end{definition}
\par \noindent \textbf{Remark} $Int(A) \subseteq A \subseteq \bar{A}$. 
\par \noindent \textbf{Remark} If $A \subseteq Y \subseteq X$, then the closure of $A$ in $Y$ and the closure of $A$ in $X$ may be different. For example let $X = \mathbb{R}$ and $Y = [0, 1)$. Let $A = (0, 1) \subseteq Y$. Then the closure of $A$ in $Y$ is $[0, 1)$ while the closure of $A$ in $X$ is $[0, 1]$. 

\begin{theorem}
    Let $Y \subseteq X$ with the subspace topology. Let $\bar{A}$ be the closure of $A$ in $X$. Then the closure of $A$ in $Y$ is equal to $\bar{A} \cap Y$.
\end{theorem}
\par \noindent \textbf{Proof} 
\par Let $B$ be the closure of $A$ in $Y$. We prove the two inclusions. 
\par Let $B$ be the closure of $A$ in $Y$. $\bar{A}$ is closed in $X$. So $\bar{A} \cap Y$ is closed in $Y$. Then $\bar{A} \cap Y \supseteq A$. So $B \subseteq \bar{A} \cap Y$. 
\par Since $B$ is closed in $Y$, there exists a closed set $C$ in $X$ such that \(B = C \cap Y\). Hence $C \supseteq A$ which is closed in $X$. So $C \supseteq \bar{A}$. Then $\bar{A} \cap Y \subseteq C \cap Y = B$. 
\\ \null \hfill $\blacksquare$  

\end{document}