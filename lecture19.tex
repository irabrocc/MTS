% 声明为子文件,指定主文件
\documentclass[main.tex]{subfiles}

\begin{document}
\pagestyle{plain}
\setcounter{chapter}{18}

\chapter{Later}
\label{chap:chapter19}

\begin{theorem}
    Every second countable regular space is normal.
\end{theorem}
\par \noindent \textbf{Proof} 
\par Let $X$ be a second countable regular space with $A, B \subseteq X$ closed and $A \cap B = \emptyset$.
\par For every $x \in A$, since $X$ is regular, there exist open sets $U\ni x$ such that $U\cap B = \emptyset$. 
\par By the lemma, there exists a open set $V$ such that $x \in V$, $\bar{V} \subseteq U$ and thus $\bar{V} \cap B = \emptyset$. 
\par Let $\mathcal{B}$ be a countable baisis for $X$. For each $x \in A$, there exists $U_x \in \mathcal{B}$ such that $x \in U_x \subseteq V$, and thus $\bar{U_x} \subseteq \bar{V} \subseteq U$ and $\bar{U_x} \cap B = \emptyset$. Since $\mathcal{B}$ is countable, the collection $\{U_x: x \in A\}$ has a countable subcollection $\{U_n: n \in \mathbb{N}\}$ covering $A$ such that $\bar{U_n} \cap B = \emptyset$ for each $n$. 

\par Similarly, we can choose a countable collection $\{V_m: m \in \mathbb{N}\}$ of open sets covering $B$, such that $\bar{V_m} \cap A = \emptyset$ for each $m$. Then the sets $U = \bigcup_{n} U_n$ and $V = \bigcup_{m} V_m$ are open sets containing $A$ and $B$ respectively. But they need not be disjoint. We perform the following simple trick to construct two open sets that are disjoint. Given $n$, define 
\begin{equation}
    U_n' = U_n \setminus \bigcup_{m=1}^{n} \bar{V_m}, \quad V_n' = V_n \setminus \bigcup_{k=1}^{n} \bar{U_k}
\end{equation} 
Note that $U_n'$ and $V_n'$ are open. The collection $\{U_n': n \in \mathbb{N}\}$ covers $A$ because each $x$ in $A$ belongs to some $U_n$ and $x$ belongs to none of the $\bar{V_m}$'s. Similarly, $\{V_n': n \in \mathbb{N}\}$ covers $B$. 
\par Finally, the open sets $U' = \bigcup_n U_n'$ and $V' = \bigcup_m V_m'$ are disjoint. For if $x \in U' \cap V'$, then there exist $n, m$ such that $x \in U_n'$ and $x \in V_m'$. Without loss of generality, assume that $n \leq m$. Then
\begin{equation}
    x \in U_n' \subseteq U_n \setminus \bigcup_{k=1}^{n} \bar{V_k} \subseteq U_n \setminus \bar{V_m}
\end{equation}
This contradicts the fact that $x \in V_m' \subseteq V_m \subseteq \bar{V_m}$. Thus, $U' \cap V' = \emptyset$. Hence, $X$ is normal.
\mbox{} \\ \null \hfill $\blacksquare$ 
\begin{theorem}
    Every metrizable space $X$ is normal.
\end{theorem}

\par \noindent \textbf{Proof} Let $d$ be the metric on $X$. Given two closed disjoint subsets $A, B \subseteq X$. 
\par For each $a\in A$, there exists $r_a > 0$ such that $B(a, r_a) \cap B = \emptyset$. 
\par Similarly, for each $b \in B$, there exists $r_b > 0$ such that $B(b, r_b) \cap A = \emptyset$.
\par Let $U = \bigcup_{a \in A} B(a, \frac{r_a}{2})$ and $V = \bigcup_{b \in B} B(b, \frac{r_b}{2})$. Then $U$ and $V$ are open sets containing $A$ and $B$ respectively. 
\par We claim that $U \cap V = \emptyset$. Assume the contrary that there exists $x \in U \cap V$. Then there exist $a \in A$ and $b \in B$ such that $x \in B(a, \frac{r_a}{2})$ and $x \in B(b, \frac{r_b}{2})$. Without loss of generality, assume that $\frac{r_a}{2} \leq \frac{r_b}{2}$. Then
\begin{equation}
    d(a, b) \leq d(a, x) + d(x, b) < \frac{r_a}{2} + \frac{r_b}{2} \leq r_a
\end{equation}
This implies that $b \in B(a, r_a)$, which contradicts the choice of $r_a$. Thus, $U \cap V = \emptyset$. Hence, $X$ is normal. 
\mbox{} \\ \null \hfill $\blacksquare$ 
\begin{theorem}
    Every compact Hausdorff space is normal.
\end{theorem}
\par \noindent \textbf{Proof} 
\par Let $x\in X$ and $B \subseteq X$ be closed such that $x \notin B$.
\par For each $y\in B$, since $X$ is Hausdorff, there exist disjoint open sets $U_y$ and $V_y$ such that $x \in U_y$ and $y \in V_y$. Then $\{V_y | y \in B\}$ is an open covering of $B$. Since $X$ is compact and Hausdorff, $B$ is compact as a closed subset of $X$. By the compactness of $B$, there exists a finite subcollection $\{V_{y_1}, V_{y_2}, \ldots, V_{y_n}\}$ that covers $B$. 
Let $V = \bigcup V_{y_i}$ and $U = \bigcap U_{y_i}$. Then $U$ and $V$ are disjoint open sets containing $x$ and $B$ respectively. So $X$ is regular. 
\par Let $A, B \subseteq X$ be closed and $A \cap B = \emptyset$. For each $a\in A$, since $X$ is regular, there exist disjoint open sets $U_a$ and $V_a$ such that $a \in U_a$ and $B \subseteq V_a$. Then $\{U_a | a \in A\}$ is an open covering of $A$. By the compactness of $A$, there exists a finite subcollection $\{U_{a_1}, U_{a_2}, \ldots, U_{a_m}\}$ that covers $A$. 
Let $U = \bigcup U_{a_i}$ and $V = \bigcap V_{a_i}$. Then $U$ and $V$ are disjoint open sets containing $A$ and $B$ respectively. So $X$ is normal. 
\mbox{} \\ \null \hfill $\blacksquare$ 

\begin{lemma}
    [Urysohn's lemma]
    Given a normal space $X$ and two closed disjoint subsets $A, B \subseteq X$, there exists a continuous map $f: X \to [0, 1]$ such that $f(A) = \{0\}$ and $f(B) = \{1\}$.
\end{lemma}
\par \noindent \textbf{Proof} The proof is too long to be included here. Please refer to Munkres' Topology, PDF P 224. It is not required to be reproduced. 
\mbox{} \\ \null \hfill $\blacksquare$ 

\end{document}