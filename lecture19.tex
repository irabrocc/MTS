% 声明为子文件,指定主文件
\documentclass[main.tex]{subfiles}

\begin{document}
\pagestyle{plain}
\setcounter{chapter}{18}

\chapter{19}
\label{chap:chapter19}

\begin{definition}
    Matrices $A\in \mathbb{C}^{n\times n}$ and $B\in \mathbb{C}^{m\times m}$ are called similar if there exists an invertible matrix $P\in \mathbb{C}^{n\times m}$ such that
    \begin{equation}
        A = PBP^{-1}
    \end{equation}
    We write $A \sim B$.
\end{definition}

\begin{property}
    Similarity is an equivalence relation on the set of all square matrices.
\end{property}

\begin{theorem}
    If $A \sim B$, then $A$ and $B$ have the same characteristic polynomial, and thus the same eigenvalues with the same algebraic multiplicities.
\end{theorem}
\par \noindent \textbf{Proof} 
\par We have 
\begin{equation}
    P(\lambda) = \det(\lambda I - A) = \det(\lambda I - PBP^{-1}) = \det(P(\lambda I - B)P^{-1}) = \det(\lambda I - B)
\end{equation}
\mbox{} \\ \null \hfill $\blacksquare$

\section{Schur's decomposition} 
\par \noindent \textbf{Idea} The idea of diagonalization may be problematic as: 

\begin{itemize}
    \item Diagonalization is not always possible. 
    \item It is neccessary to compute the inverse of $P$. 
    \item Computing the decompostition is much easier if $P$ is unitary(called orthogonal if real).
\end{itemize}

\begin{theorem}
    [Shur's decomposition]
    For all $A \in \mathbb{C}^{n \times n}$, there exists an upper triangular $T\in \mathbb{C}^{n \times n}$ and a unitary $U \in \mathbb{C}^{n \times n}$ such that
    \begin{equation}
        A = UTU^*
    \end{equation}
\end{theorem}
\par \noindent \textbf{Proof} 
\par Let $\lambda_1$ be an eigenvalue of $A$. We choose orthogonal bases $\{q_1, q_2, \ldots, q_k\} = E_{\lambda_1}$ of the eigenspace of $\lambda_1$ and $E_{\lambda_1}^{\perp}$ respectively. Let $Q_1 = [q_1, q_2, \ldots, q_n]$ and thus $Q_1$ is unitary. 
We obtain 
\begin{equation}
    A = Q_1 \begin{bmatrix}
        \lambda_1 I_k & B_1 \\
        0 & A_1
    \end{bmatrix} Q_1^*
\end{equation}
\par Repeating the process on $A_1$, we obtain a unitary $Q_2$ such that
\begin{equation}
    A_1 = Q_2 \begin{bmatrix}
        \lambda_2 I_{k_2} & B_2 \\
        0 & A_2
    \end{bmatrix} Q_2^*
\end{equation}
\par Continuing this process, we eventually obtain
\begin{equation}
    A = U T U^*
\end{equation}
where $U = Q_1 Q_2 \cdots Q_m$ is unitary
and $T$ is upper triangular.
\mbox{} \\ \null \hfill $\blacksquare$

\begin{definition}
    A matrix $A\in \mathbb{C}^{n\times n}$ is called normal if $AA^* = A^*A$.
\end{definition}
\par \noindent \textbf{Remark} A Hermitian matrix is normal since $A^*A = A^2 = AA^*$. A unitary matrix is normal since $AA^* = I = A^*A$. 

\begin{theorem}
    A matrix $A\in \mathbb{C}^{n\times n}$ is normal if and only if it is unitarity diagonalizable, i.e., there exists a unitary matrix $U$ and a diagonal matrix $\Lambda$ such that
    \begin{equation}
        A = U \Lambda U^*
    \end{equation}
\end{theorem}

\par \noindent \textbf{Proof} 
\par Suppose $A = U T U^*$ is the Schur decomposition of $A$. If $A$ is normal, then
\begin{equation}
    (UTU^*)(UTU^*)^* = (UTU^*)^*(UTU^*)
\end{equation}
\par This implies that $TT^* = T^*T$. Since $T$ is upper triangular, $T$ must be diagonal. Thus, $A$ is unitarily diagonalizable.
\mbox{} \\ \null \hfill $\blacksquare$ 

\par \noindent \textbf{Remark} A matrix that is diagonalizable but not unitarily diagonalizable is
\begin{equation}
    A = \begin{bmatrix}
        1 & 1 \\
        0 & 2
    \end{bmatrix}
\end{equation}
The eigenvalues of $A$ are $1$ and $2$, but $AA^* \neq A^*A$. (Exercise)

\begin{corollary}
    For a normal matrix, eigenvectors corresponding to distinct eigenvalues are orthogonal, or can be normalized to be orthonormal. The eigenvectors related to the same eigenvalue can be orthogonalized by using Gram-Schmidt process. 
\end{corollary}

\section{Singular Value Decomposition} 
\begin{theorem}
    [Singular Value Decomposition]
    For any matrix $A \in \mathbb{C}^{m \times n}$, it can be represented in the form 
    \begin{equation}
        A = U \Sigma V^*
    \end{equation}
    where $U \in \mathbb{C}^{m \times p}$ and $V \in \mathbb{C}^{n \times p}$ have orthonormal columns(so $U$ is unitary if $m = p$ and $V$ is unitary if $n = p$), $\Sigma \in \mathbb{C}^{p \times p}$ satisfies
    \begin{equation}
        \Sigma = \begin{bmatrix}
            \sigma_1 & 0 & \cdots & 0 \\
            0 & \sigma_2 & \cdots & 0 \\
            \vdots & \vdots & \ddots & \vdots \\
            0 & 0 & \cdots & \sigma_p \\
        \end{bmatrix}
    \end{equation}
    where $p = \min\{m, n\}$, such that $\sigma_1 \geq \sigma_2 \geq \cdots \geq \sigma_p \ge 0$ are the singular values of $A$ .
\end{theorem}




\end{document}