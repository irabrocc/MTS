% 声明为子文件,指定主文件
\documentclass[main.tex]{subfiles}

\begin{document}
\pagestyle{plain}
\setcounter{chapter}{21}

\chapter{Lecture 22}
\label{chap:chapter22}

\begin{definition}
    A compactification of a Hausdorff space $X$ is a Hausdorff space $Y \supseteq X$ such that $Y = \overline{X}$. We say that compactifications $Y_1$ and $Y_2$ are equivalent if there exists a homeomorphism $h: Y_1 \to Y_2$ such that $h|_{X} = id_{X}$.
\end{definition}
\begin{example}
    $X = (0, 1)$, $Y = S^1 \subseteq \mathbb{R}^2$, $X\ni x \mapsto (\cos 2\pi x, \sin 2\pi x) \in Y$ is a compactification of $X$. 
\end{example}
\begin{example}
    $X = (0, 1)$, $Y = [0, 1] \subseteq \mathbb{R}$. 
\end{example}

\begin{property}
    Let $X \subseteq Y$ and $Y$ be a compact Hausdorff space. Then $X$ is completely regular. 
\end{property}
\par \noindent \textbf{Proof} 
\par Since $Y$ is compact Hausdorff, $Y$ is normal. Then $Y$ is completely regular. Thus $X$ is completely regular as well. \mbox{} \\ \null \hfill $\blacksquare$

\par \noindent \textbf{Claim}
    If $X$ is completely regular, it has a compactification.

\begin{theorem}
    Let $X$ be $T1$ space. Suppose $X$ has an indexed family of continuous functions $\{f_{\alpha}: X \rightarrow \mathbb{R}\}_{\alpha \in J}$ such that $\forall x_0 \in X$, and $U \ni x_0$ open, there exists a function $f_{\alpha}$ such that $f_{\alpha}(x_0) = 1$ and $f_{\alpha}(X \setminus U) \equiv 0$. Then $F : X \rightarrow \mathbb{R}^J$ defined by $F(x) = (f_{\alpha}(x))_{\alpha \in J}$ is an embedding $X \hookrightarrow \mathbb{R}^J$. If, additionally, $f_{\alpha}: X \rightarrow [0, 1]$ for all $\alpha \in J$, then $F$ is an embedding $X \hookrightarrow [0, 1]^J$.  
\end{theorem}
\par \noindent \textbf{Proof} To be done. 

\begin{theorem}
    $X$ is completely regular if and only if it is homeomorphic to a subspace of a cube $[0, 1]^J$ for some index set $J$.
\end{theorem}
\par \noindent \textbf{Proof} Not shown in the lecture. 

\begin{lemma}
    Let $X$ be a Hausdorff space. Let $h: X\hookrightarrow Z$ be an embedding with $Z$ compact Hausdorff. Then there exists a compacdification $Y$ of $X$ such that there is an embedding $H: Y \hookrightarrow Z$ with $H|_{X} = h$. Such a compactification is uniquely determined. 
\end{lemma}

\begin{example}
    Let $X = (0, 1)$. Consider the embedding $h: X \hookrightarrow \mathbb{R}^2$ defined by $h(x) = (x, \sin \frac{1}{x})$. Let $A = \{0\} \times [-1, 1] \cup \{(1, \sin(1))\}$. Then $Y = h(X) \cup A$ is a compactification of $h(X)$.
\end{example}

\par There's something. 

\begin{theorem}
    [Stone–Čech compactification]
    Let $X$ be a completely regular space. Then there exists a compactification $Y$ of $X$ such that every bounded continuous function $f: X \rightarrow \mathbb{R}$ can be extended uniquely to a continuous function $\overline{f}: Y \rightarrow \mathbb{R}$. ($Y$ is called the Stone–Čech compactification of $X$) 
\end{theorem}
\par \noindent \textbf{Proof} To be done. 


\end{document}